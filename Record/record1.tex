\documentclass{jlreq}
\usepackage{silence}
\WarningFilter{caption}{Unknown document}
\usepackage{fancyhdr,lastpage,abstract,listings,float,wrapfig,tabularx,multirow,multicol}
\usepackage{hyperref,graphics,graphicx,framed}
\usepackage[symbol]{footmisc}
\usepackage{amsmath,amssymb}
\usepackage{tikz}
\usetikzlibrary{calc,arrows.meta,fit,positioning,shapes}
\newcolumntype{A}{>{\centering\bfseries}X}
\newcolumntype{B}{>{\centering\bfseries\arraybackslash}X}
\newcolumntype{C}{>{\centering\arraybackslash}X}
\newcolumntype{R}{>{\raggedright\arraybackslash}X}
\newcolumntype{L}{>{\raggedleft\arraybackslash}X}
\bibliographystyle{junsrt}
\hypersetup{
    colorlinks=true,
    citecolor=black,
    linkcolor=black,
    urlcolor=black
}
\renewcommand{\figurename}{図}
\renewcommand{\tablename}{表}
\renewcommand{\lstlistingname}{src.}
\renewcommand{\thefootnote}{\fnsymbol{footnote}}
\renewcommand{\contentsname}{目次}
\renewcommand{\listfigurename}{図目次}
\renewcommand{\listtablename}{表目次}
\renewcommand{\lstlistlistingname}{ソースコード}
\renewcommand{\thefigure}{\arabic{figure}}
\renewcommand{\thetable}{\arabic{table}}
\renewcommand{\theequation}{\arabic{equation}}
\renewcommand{\labelenumi}{\theenumi.\ }
\renewcommand{\eqref}[1]{式(\ref{#1})}
\newcommand{\figref}[1]{図\ref{#1}}
\newcommand{\tblref}[1]{表\ref{#1}}
\newcommand{\srcref}[1]{src.\ref{#1}}
\newcommand{\chapref}[1]{\ref{#1}節}
\makeatletter
\@addtoreset{footnote}{page}
\makeatother
\AtBeginDocument{
    \renewcommand{\thelstlisting}{\arabic{lstlisting}}
    \maketitle
}
\newcommand\YAMLcolonstyle{\color{red}\mdseries}
\newcommand\YAMLkeystyle{\color{black}\bfseries}
\newcommand\YAMLvaluestyle{\color{blue}\mdseries}

\makeatletter

% here is a macro expanding to the name of the language
% (handy if you decide to change it further down the road)
\newcommand\language@yaml{yaml}

\expandafter\expandafter\expandafter\lstdefinelanguage
\expandafter{\language@yaml}
{
keywords={true,false,null,y,n},
sensitive=false,
comment=[l]{\#},
morecomment=[s]{/*}{*/},
moredelim=[l][\color{orange}]{\&},
moredelim=[l][\color{magenta}]{*},
moredelim=**[il][\YAMLcolonstyle{:}\YAMLvaluestyle]{:},   % switch to value style at :
morestring=[b]',
morestring=[b]",
literate =    {---}{{\ProcessThreeDashes}}3
{>}{{\textcolor{red}\textgreater}}1
{|}{{\textcolor{red}\textbar}}1
{\ -\ }{{\mdseries\ -\ }}3,
}

% switch to key style at EOL
\lst@AddToHook{EveryLine}{\ifx\lst@language\language@yaml\YAMLkeystyle\fi}
\makeatother

\newcommand\ProcessThreeDashes{\llap{\color{cyan}\mdseries-{-}-}}

\lstset{
    %backgroundcolor={\color[gray]{.90}},
    breaklines = true,
    breakindent = 10pt,
    basicstyle = \ttfamily\small,
    commentstyle = {\ttfamily \color[rgb]{0,0.45,0}},
    classoffset = 0,
    keywordstyle = {\bfseries \color{red}},
    stringstyle = {\ttfamily \color{blue}},
    frame = single,
    %他オプション:leftline,topline,bottomline,lines,single,shadowbox
    framesep = 5pt,
    numbers = left,
    stepnumber = 1,
    numberstyle = \small,
    tabsize = 4,
    captionpos = t,
    otherkeywords = {},
    xleftmargin = 2em
}
\usepackage[margin=20truemm,headheight=20pt]{geometry}
\usepackage{fancyhdr,lastpage}
\fancypagestyle{fancy}{
    \fancyhf{}
    \fancyhead[R]{\thepage\ /\ \pageref{LastPage}}
    \fancyhead[C]{作業記録(Group C5)}
    \fancyhead[L]{July 13th, 2023}
}
\pagestyle{fancy}
\begin{document}
\begin{framed}
    \tableofcontents
\end{framed}
\section{Webシステム}
\subsection{HTTP通信}
\newcommand{\apache}{\texttt{Apache}}
\begin{enumerate}
    \item \apache をインストールする.
          \begin{lstlisting}
$ sudo su %\enter%
# apt update %\enter%
# apt install apache2 %\enter%
    \end{lstlisting}
    \item Baisc認証を設定する.IDを\texttt{testuser},パスワードを\texttt{root00}とする.
          \begin{lstlisting}
# htpasswd -c /etc/apache2/.htpasswd testuser %\enter%
New password: root00 %\enter%
Re-type new password: root00 %\enter%
Adding password for user testuser
\end{lstlisting}
    \item \texttt{/var/www/html/}の下にある\texttt{index.html}の名前を変更し,新しく\texttt{index.html}を作成する.
          \begin{lstlisting}[style=file,caption={\ttfamily /var/www/html/index.html},language=html]
<h1>Group 5C</h1>
<h2>/var/www/html/index.html</h2>
    \end{lstlisting}
    \item \texttt{/var/www/html/basic/}と\texttt{/var/www/html/inside/}に\texttt{index.html}を作成する.
          \begin{lstlisting}[style=file,language=html,caption={\ttfamily /var/www/html/basic/index.html}]
<h1>Group 5C</h1>
<h2>/var/www/html/basic/index.html</h2>
\end{lstlisting}
          \begin{lstlisting}[style=file,language=html,caption={\ttfamily /var/www/html/inside/index.html}]
<h1>Group 5C</h1>
<h2>/var/www/html/inside/index.html</h2>
    \end{lstlisting}
    \item \texttt{/var/www/html/basic/}のディレクトリにBasic認証を適用する.(以下を追記する.)
          \begin{lstlisting}[style=file,caption={\ttfamily /etc/apache2/sites-available/000-default.conf},label={src:basic}]
<Directory /var/www/html/basic>
    AuthType Basic
    AuthUserFile /etc/apache2/.htpasswd
    AuthName "basic auth"
    Satisfy any
    Order deny,allow
    Deny from all
    Require valid-user
</Directory>
\end{lstlisting}
    \item \texttt{/var/www/html/inside/}ディレクトリにIPアドレスによるアクセス制限を設ける.(以下を追記する.)\texttt{192.168.0.161},\texttt{192.168.0.162}以外からのアクセスを拒否する.
          \begin{lstlisting}[style=file,caption={\ttfamily /etc/apache2/sites-available/000-default.conf},label={src:inside}]
<Directory /var/www/html/inside>
    Order deny,allow
    Deny from all
    Allow from 192.168.0.161
    Allow from 192.168.0.162
</Directory>
    \end{lstlisting}
    \item \apache を再起動する.
          \begin{lstlisting}
# systemctl restart apache2 %\enter%
\end{lstlisting}
\end{enumerate}
\subsection{HTTPS通信と証明書}
\begin{enumerate}
    \item Ubuntu の\apache で\texttt{SSL/TLS}通信をするために必要なモジュールを使用する設定.
          \begin{lstlisting}
# a2emod ssl %\enter%
# a2ensite default-ssl %\enter%
# systemctl restart apache2 %\enter%
\end{lstlisting}
    \item \srcref{src:basic},\srcref{src:inside}の内容を\texttt{/etc/apache2/sites-available/default-ssl.conf}にも追記する.
    \item 秘密鍵を作成する.
          \begin{lstlisting}
# openssl genrsa -out server.key 2048 %\enter%
\end{lstlisting}
    \item 秘密鍵\texttt{server.key}に対応する証明書署名要求\texttt{CSR}を作成する.
          \begin{lstlisting}
# openssl req -new -key server.key -sha256 -out server.CSR %\enter%
Country Name: JP %\enter%
State or Privience Name: Kochi %\enter%
Locality Name: Kochi-shi %\enter%
Organization Name: Kochi Purefectual Public University Corporation %\enter%
Organization Unit Name: . %\enter%
Common Name: www.c5.exp.info.kochi-tech.ac.jp %\enter%
Email Address: . %\enter%
extra: %\enter% %\enter%
# 
\end{lstlisting}
    \item 作成した証明書をTAに渡す.
    \item 自分の秘密鍵\texttt{server.key}で署名する.証明書ファイルは\texttt{server.crt}.
          \begin{lstlisting}
# openssl x509 -req -days 365 -in server.csr -signkey server.key -out server.crt %\enter%
\end{lstlisting}
    \item 証明書ファイルを設置する.
          \begin{lstlisting}
# mv server.key /etc/ssl/private/ %\enter%
# mv server.crt /etc/ssl/certs/ %\enter%
\end{lstlisting}
          \begin{lstlisting}[style=file,caption={\ttfamily /etc/apache2/sites-available/default-ssl.conf}]
SSLCertificateFile /etc/ssl/certs/server.crt
SSLCertificateKeyFile /etc/ssl/private/server.key
\end{lstlisting}
    \item 鍵の権限を変更する.
          \begin{lstlisting}
# chown root:root /etc/ssl/certs/server.crt %\enter%
# chmod 644 /etc/ssl/certs/server.crt %\enter%
# chown root:ssl-cert /etc/ssl/private/server.key %\enter%
# chmod 640 /etc/ssl/private/server.key %\enter%
\end{lstlisting}
    \item \apache を再起動する.
\end{enumerate}
\subsection{動的コンテンツの設定}
\begin{enumerate}
    \item \apache でCGIを利用できるようにする.
          \begin{lstlisting}
# a2enmod cgi %\enter%
\end{lstlisting}
          以下のコメントアウトを解除する.
          \begin{lstlisting}[style=file,caption={\ttfamily /etc/apache2/mods-enabled/mime.conf}]
AddHandler cgi-script .cgi
\end{lstlisting}
          \texttt{/var/www/html/cig/}ディレクトリのCGIを許可する.(以下を追記する.)
          \begin{lstlisting}[style=file,caption={\ttfamily /etc/apache2/sites-available/default-ssl.conf}]
<Directory /var/www/html/cgi>
    Options +ExecCGI
</Directory>
\end{lstlisting}
    \item \apache でSSIを利用できるようにする.
          \begin{lstlisting}
# a2enmod include %\enter%
\end{lstlisting}
          以下の2行が有効であることを確認する.
          \begin{lstlisting}[style=file,caption={\ttfamily /etc/apache2/mods-enabled/mime.conf}]
AddType text/html .shtml
AddOutputFilter INCLUDES .shtml
\end{lstlisting}
          \texttt{/var/www/html/ssi/}ディレクトリのSSIを許可する.
          \begin{lstlisting}[style=file,caption={\ttfamily /etc/apache2/sites-available/default-ssl.conf}]
<Directory /var/www/html/ssi>
    Options Includes
</Directory>
\end{lstlisting}
    \item \apache を再起動する.
    \item PHPと\apache のPHP呼び出しモジュールをインストールする.
          \begin{lstlisting}
# apt install php
# apt install libapache2-mod-php
\end{lstlisting}
    \item PHPプログラムを取得する.
          \begin{lstlisting}
# wget ftp://222.229.69.11/00/time.cgi %\enter%
# wget ftp://222.229.69.11/00/time.shtml %\enter%
# wget ftp://222.229.69.11/00/counter.cgi %\enter%
# wget ftp://222.229.69.11/00/time.php
\end{lstlisting}
    \item これらを,適切な場所に配置する.
          \begin{lstlisting}
# mkdir /var/www/html/cgi/ %\enter%
# mv time.cgi /var/www/html/cgi/ %\enter%
# mkdir /var/www/html/ssi %\enter%
# mv time.shtml /var/www/html/ssi/ %\enter%
# mv counter.cgi /var/www/html/cgi/ %\enter%
\end{lstlisting}
    \item \texttt{time.cgi},\texttt{counter.cgi}に実行権限を与える.
          \begin{lstlisting}
# chmod 755 /var/www/html/cgi/time.cgi %\enter%
# chmod 755 /var/www/html/cgi/counter.cgi %\enter%
\end{lstlisting}
    \item \texttt{counter.cgi}で用いるカウンタファイルを作成する.
          \begin{lstlisting}
# echo "0" > /var/www/html/cgi/count.dat %\enter%
# chmod 644 count.dat %\enter%
# chown www-data:www-data count.dat %\enter%
\end{lstlisting}
\end{enumerate}
\clearpage
\section{データベース・Webシステム}
\subsection{Windowsの端末ソフトウェア}
\begin{enumerate}
    \item \texttt{PuTTY}をインストールする.
    \item \texttt{PuTTY}の文字コードを\texttt{UTF-8}へ変更する.\\
          \texttt{Window → Translation → Remote character set}で\texttt{UTF-8}とする.
    \item SSHでサーバへログインする.
    \item 日本語文字コード,改行コードの変換に使われる\texttt{NKF}をインストールする.
          \begin{lstlisting}
# apt update %\enter%
# apt install nkf %\enter%
\end{lstlisting}
\end{enumerate}
\subsection{MySQLのインストールと文字コードの設定}
\begin{enumerate}
    \item シェルの文字コードを\texttt{UTF-8}に設定する.
          \begin{lstlisting}
$ sudo su %\enter%
# locale-gen ja_JP.UTF-8 %\enter%
\end{lstlisting}
          以下を追記する.
          \begin{lstlisting}[style=file,caption={\ttfamily \~\ /.bashrc}]
export LANG=ja_JP.UTF-8
\end{lstlisting}
    \item MySQLをインストール
          \begin{lstlisting}
# apt install mysql-server %\enter%
\end{lstlisting}
    \item MySQLの文字コードを\texttt{UTF-8}に変更.(以下を追記する.)
          \begin{lstlisting}[style=file,caption={\ttfamily /etc/mysql/conf.d/mysql.cnf}]
default-character-set=utf8
\end{lstlisting}
          \begin{lstlisting}[style=file,caption={\ttfamily /etc/mysql/mysql.conf.d/mysqld.cnf}]
character-set-server=utf8    
\end{lstlisting}
    \item MySQLを再起動する.
          \begin{lstlisting}
# systemctl restart mysql %\enter%
\end{lstlisting}
\end{enumerate}
\subsection{データベースの構築}
\begin{enumerate}
    \item FTPで3つの\texttt{CSV}ファイルを取得する.
          \begin{lstlisting}
ftp 222.229.69.11 %\enter%
    Connected to 222.229.69.11.
    220 mainserver FTP server (Version 6.00LS) ready.
Name (222.229.69.11:exp): ftp %\enter%
    331 Guest login ok, send your email address as password.
Password: %\enter%
ftp> get bunrui.csv %\enter%
ftp> get tanka.csv %\enter%
ftp> get uriage.csv %\enter%
ftp> quite %\enter%
221 Goodbey.
\end{lstlisting}
    \item 取得した\texttt{CSV}ファイルを\texttt{UTF-8}に変換し,Index(1行目)を削除する.
          \begin{lstlisting}
# nkf -w -d bunrui.csv > bunrui-utf8.csv %\enter%
# nkf -w -d tanka.csv > tanka-utf8.csv %\enter%
# nkf -w -d uriage.csv > uriage-utf8.csv %\enter%
\end{lstlisting}
    \item サーバ側でのローカルファイルからのデータ一括入力の可否を制御する変数\texttt{local infile}を\texttt{1}(可)に変更する.
          \begin{lstlisting}
# mysql -u root %\enter%
mysql> set persist local_infile=1; %\enter%
\end{lstlisting}
    \item ラーメン屋売上データベースを作成する.
          \begin{lstlisting}
# mysql -u root -p %\enter%
mysql> create database sales; %\enter%
mysql> use sales; %\enter%
mysql> create table sales ( %\enter%
    -> date text, %\enter%
    -> food text, %\enter%
    -> amount integer ); %\enter%
mysql> create table price ( %\enter%
    -> food text, %\enter%
    -> fee integer ); %\enter%
mysql> create table foodgroup ( %\enter%
    -> food text, %\enter%
    -> foodgroup text ); %\enter%
mysql> exit %\enter%
\end{lstlisting}
    \item \texttt{CSV}ファイルを読み込む.
          \begin{lstlisting}
# mysql -u root --local-infile=1 -p %\enter%
mysql> use sales; %\enter%
mysql> load data local infile "uriage-utf8.csv" into table sales fields terminated by ','; %\enter%
mysql> load data local infile "tanka-utf8.csv" into table price fields terminated by ','; %\enter%
mysql> load data local infile "bunrui-utf8.csv" into table foodgroup fields terminated by ','; %\enter%
\end{lstlisting}
\end{enumerate}
\subsection{データベースの操作}
\begin{enumerate}
    \item データベースにアクセスする.
          \begin{lstlisting}
# mysql -u root -p %\enter%
mysql> use sales; %\enter%
    \end{lstlisting}
    \item 2006/12/31 のサイダー2本を削除.
          \begin{lstlisting}
mysql> delete from sales.sales where sales.date='2006/12/31' and sales.food='サイダー'; %\enter%
\end{lstlisting}
    \item 2006/12/31 へオレンジジュース2本を追加.
          \begin{lstlisting}
mysql> insert into sales values ('2006/12/31', 'オレンジジュース', 2); %\enter%
\end{lstlisting}
    \item テーブル\texttt{sales}から2006/12/26に売り上げた食品の品目と個数を表示する(日付の表示はしない).
          \begin{lstlisting}
mysql> select food,amount from sales where date='2006/12/26'; %\enter%
+--------------------------+--------+
| food                     | amount |
+--------------------------+--------+
| 味噌ラーメン             |     11 |
| ラーメン                 |     34 |
| タンメン                 |     11 |
| もやしラーメン           |      8 |
| 五目ラーメン             |     21 |
| とんこつラーメン         |     17 |
| チャーシューメン         |     11 |
| 味噌チャーシュー         |      9 |
| ねぎラーメン             |     35 |
| 味噌ねぎラーメン         |      7 |
| チャーハン               |     19 |
| 五目チャーハン           |     13 |
| 海老チャーハン           |      4 |
| 中華飯                   |     14 |
| 餃子                     |     21 |
| 野菜餃子                 |     18 |
| シュウマイ               |      5 |
| 春巻き                   |      5 |
| オレンジジュース         |      1 |
| ウーロン茶               |      2 |
| オレンジジュース         |      2 |
+--------------------------+--------+
21 rows in set (0.00 sec)
\end{lstlisting}
    \item テーブル\texttt{sales}の食品名の横に,テーブル\texttt{foodgroup}の分類名を参照して,日付,食品名,分類,個数を表示する.
          \begin{lstlisting}
mysql> select sales.date, sales.food, foodgroup.foodgroup, sales.amount from sales,foodgroup where sales.food=foodgroup.food %\enter%
+------------+--------------------------+--------------+--------+
| date       | food                     | foodgroup    | amount |
+------------+--------------------------+--------------+--------+
| 2006/12/25 | 味噌ラーメン             | 麺           |     21 |
| 2006/12/25 | ラーメン                 | 麺           |     36 |
| 2006/12/25 | タンメン                 | 麺           |     17 |
| 2006/12/25 | もやしラーメン           | 麺           |     14 |
| 2006/12/25 | 五目ラーメン             | 麺           |     30 |
| 2006/12/25 | とんこつラーメン         | 麺           |     39 |
| 2006/12/25 | チャーシューメン         | 麺           |     11 |
| 2006/12/25 | 味噌チャーシュー         | 麺           |      8 |
| 2006/12/25 | ねぎラーメン             | 麺           |     24 |
| 2006/12/25 | 味噌ねぎラーメン         | 麺           |      5 |
| 2006/12/25 | チャーハン               | ご飯         |     16 |
| 2006/12/25 | 五目チャーハン           | ご飯         |      8 |
| 2006/12/25 | 海老チャーハン           | ご飯         |      3 |
| 2006/12/25 | 中華飯                   | ご飯         |     11 |
| 2006/12/25 | 餃子                     | 飲茶         |     56 |
| 2006/12/25 | 野菜餃子                 | 飲茶         |     28 |
| 2006/12/25 | シュウマイ               | 飲茶         |      7 |
| 2006/12/25 | 春巻き                   | 飲茶         |      9 |
| 2006/12/25 | オレンジジュース         | ドリンク     |      4 |
| 2006/12/25 | ウーロン茶               | ドリンク     |      8 |
| 2006/12/25 | サイダー                 | ドリンク     |      1 |
| 2006/12/26 | 味噌ラーメン             | 麺           |     32 |
| 2006/12/26 | ラーメン                 | 麺           |     45 |
| 2006/12/26 | タンメン                 | 麺           |     11 |
| 2006/12/26 | もやしラーメン           | 麺           |      5 |
| 2006/12/26 | 五目ラーメン             | 麺           |     12 |
| 2006/12/26 | とんこつラーメン         | 麺           |     24 |
| 2006/12/26 | チャーシューメン         | 麺           |     28 |
| 2006/12/26 | 味噌チャーシュー         | 麺           |      8 |
| 2006/12/26 | ねぎラーメン             | 麺           |     32 |
| 2006/12/26 | 味噌ねぎラーメン         | 麺           |     10 |
| 2006/12/26 | チャーハン               | ご飯         |     14 |
| 2006/12/26 | 五目チャーハン           | ご飯         |      5 |
| 2006/12/26 | 海老チャーハン           | ご飯         |      3 |
| 2006/12/26 | 中華飯                   | ご飯         |      8 |
| 2006/12/26 | 餃子                     | 飲茶         |     72 |
| 2006/12/26 | 野菜餃子                 | 飲茶         |     11 |
| 2006/12/26 | シュウマイ               | 飲茶         |      8 |
| 2006/12/26 | 春巻き                   | 飲茶         |      3 |
| 2006/12/26 | オレンジジュース         | ドリンク     |      1 |
| 2006/12/26 | ウーロン茶               | ドリンク     |     14 |
| 2006/12/26 | サイダー                 | ドリンク     |      2 |
| 2006/12/27 | 味噌ラーメン             | 麺           |     12 |
| 2006/12/27 | ラーメン                 | 麺           |     29 |
| 2006/12/27 | タンメン                 | 麺           |     10 |
| 2006/12/27 | もやしラーメン           | 麺           |      5 |
| 2006/12/27 | 五目ラーメン             | 麺           |     38 |
| 2006/12/27 | とんこつラーメン         | 麺           |     16 |
| 2006/12/27 | チャーシューメン         | 麺           |     11 |
| 2006/12/27 | 味噌チャーシュー         | 麺           |      6 |
| 2006/12/27 | ねぎラーメン             | 麺           |     18 |
| 2006/12/27 | 味噌ねぎラーメン         | 麺           |      5 |
| 2006/12/27 | チャーハン               | ご飯         |     13 |
| 2006/12/27 | 五目チャーハン           | ご飯         |      5 |
| 2006/12/27 | 海老チャーハン           | ご飯         |      1 |
| 2006/12/27 | 中華飯                   | ご飯         |      8 |
| 2006/12/27 | 餃子                     | 飲茶         |     62 |
| 2006/12/27 | 野菜餃子                 | 飲茶         |     32 |
| 2006/12/27 | シュウマイ               | 飲茶         |      3 |
| 2006/12/27 | 春巻き                   | 飲茶         |      5 |
| 2006/12/27 | オレンジジュース         | ドリンク     |      1 |
| 2006/12/27 | ウーロン茶               | ドリンク     |      4 |
| 2006/12/27 | サイダー                 | ドリンク     |      1 |
| 2006/12/28 | 味噌ラーメン             | 麺           |     46 |
| 2006/12/28 | ラーメン                 | 麺           |     59 |
| 2006/12/28 | タンメン                 | 麺           |     30 |
| 2006/12/28 | もやしラーメン           | 麺           |     21 |
| 2006/12/28 | 五目ラーメン             | 麺           |     53 |
| 2006/12/28 | とんこつラーメン         | 麺           |     21 |
| 2006/12/28 | チャーシューメン         | 麺           |     11 |
| 2006/12/28 | 味噌チャーシュー         | 麺           |      8 |
| 2006/12/28 | ねぎラーメン             | 麺           |     42 |
| 2006/12/28 | 味噌ねぎラーメン         | 麺           |     18 |
| 2006/12/28 | チャーハン               | ご飯         |     25 |
| 2006/12/28 | 五目チャーハン           | ご飯         |     12 |
| 2006/12/28 | 海老チャーハン           | ご飯         |      8 |
| 2006/12/28 | 中華飯                   | ご飯         |     19 |
| 2006/12/28 | 餃子                     | 飲茶         |     70 |
| 2006/12/28 | 野菜餃子                 | 飲茶         |     28 |
| 2006/12/28 | シュウマイ               | 飲茶         |      3 |
| 2006/12/28 | 春巻き                   | 飲茶         |      1 |
| 2006/12/28 | オレンジジュース         | ドリンク     |      9 |
| 2006/12/28 | ウーロン茶               | ドリンク     |     14 |
| 2006/12/28 | サイダー                 | ドリンク     |      6 |
| 2006/12/29 | 味噌ラーメン             | 麺           |     34 |
| 2006/12/29 | ラーメン                 | 麺           |     47 |
| 2006/12/29 | タンメン                 | 麺           |     11 |
| 2006/12/29 | もやしラーメン           | 麺           |     28 |
| 2006/12/29 | 五目ラーメン             | 麺           |     29 |
| 2006/12/29 | とんこつラーメン         | 麺           |     35 |
| 2006/12/29 | チャーシューメン         | 麺           |     17 |
| 2006/12/29 | 味噌チャーシュー         | 麺           |     12 |
| 2006/12/29 | ねぎラーメン             | 麺           |     30 |
| 2006/12/29 | 味噌ねぎラーメン         | 麺           |     19 |
| 2006/12/29 | チャーハン               | ご飯         |     24 |
| 2006/12/29 | 五目チャーハン           | ご飯         |     12 |
| 2006/12/29 | 海老チャーハン           | ご飯         |      6 |
| 2006/12/29 | 中華飯                   | ご飯         |     17 |
| 2006/12/29 | 餃子                     | 飲茶         |     76 |
| 2006/12/29 | 野菜餃子                 | 飲茶         |     31 |
| 2006/12/29 | シュウマイ               | 飲茶         |     11 |
| 2006/12/29 | 春巻き                   | 飲茶         |      8 |
| 2006/12/29 | オレンジジュース         | ドリンク     |      2 |
| 2006/12/29 | ウーロン茶               | ドリンク     |     20 |
| 2006/12/29 | サイダー                 | ドリンク     |      5 |
| 2006/12/30 | 味噌ラーメン             | 麺           |     31 |
| 2006/12/30 | ラーメン                 | 麺           |     43 |
| 2006/12/30 | タンメン                 | 麺           |     20 |
| 2006/12/30 | もやしラーメン           | 麺           |     18 |
| 2006/12/30 | 五目ラーメン             | 麺           |     19 |
| 2006/12/30 | とんこつラーメン         | 麺           |     42 |
| 2006/12/30 | チャーシューメン         | 麺           |     17 |
| 2006/12/30 | 味噌チャーシュー         | 麺           |     21 |
| 2006/12/30 | ねぎラーメン             | 麺           |     18 |
| 2006/12/30 | 味噌ねぎラーメン         | 麺           |     18 |
| 2006/12/30 | チャーハン               | ご飯         |     31 |
| 2006/12/30 | 五目チャーハン           | ご飯         |     10 |
| 2006/12/30 | 海老チャーハン           | ご飯         |      6 |
| 2006/12/30 | 中華飯                   | ご飯         |     15 |
| 2006/12/30 | 餃子                     | 飲茶         |     67 |
| 2006/12/30 | 野菜餃子                 | 飲茶         |     26 |
| 2006/12/30 | シュウマイ               | 飲茶         |      8 |
| 2006/12/30 | 春巻き                   | 飲茶         |     18 |
| 2006/12/30 | オレンジジュース         | ドリンク     |      7 |
| 2006/12/30 | ウーロン茶               | ドリンク     |     11 |
| 2006/12/30 | サイダー                 | ドリンク     |      8 |
| 2006/12/31 | 味噌ラーメン             | 麺           |     11 |
| 2006/12/31 | ラーメン                 | 麺           |     34 |
| 2006/12/31 | タンメン                 | 麺           |     11 |
| 2006/12/31 | もやしラーメン           | 麺           |      8 |
| 2006/12/31 | 五目ラーメン             | 麺           |     21 |
| 2006/12/31 | とんこつラーメン         | 麺           |     17 |
| 2006/12/31 | チャーシューメン         | 麺           |     11 |
| 2006/12/31 | 味噌チャーシュー         | 麺           |      9 |
| 2006/12/31 | ねぎラーメン             | 麺           |     35 |
| 2006/12/31 | 味噌ねぎラーメン         | 麺           |      7 |
| 2006/12/31 | チャーハン               | ご飯         |     19 |
| 2006/12/31 | 五目チャーハン           | ご飯         |     13 |
| 2006/12/31 | 海老チャーハン           | ご飯         |      4 |
| 2006/12/31 | 中華飯                   | ご飯         |     14 |
| 2006/12/31 | 餃子                     | 飲茶         |     21 |
| 2006/12/31 | 野菜餃子                 | 飲茶         |     18 |
| 2006/12/31 | シュウマイ               | 飲茶         |      5 |
| 2006/12/31 | 春巻き                   | 飲茶         |      5 |
| 2006/12/31 | オレンジジュース         | ドリンク     |      1 |
| 2006/12/31 | ウーロン茶               | ドリンク     |      2 |
| 2006/12/31 | オレンジジュース         | ドリンク     |      2 |
+------------+--------------------------+--------------+--------+
147 rows in set (0.00 sec)
\end{lstlisting}
    \item テーブル\texttt{sales}から,2006/12/30に売り上げたもののうち,麺でかつ単価が750円以上のものについて,売り上げ日,食品名,個数を表示する.
          \begin{lstlisting}
mysql> select date,sales.food,amount,foodgroup from sales,foodgroup,price where sales.food=foodgroup.food and sales.food=price.food and foodgroup.foodgroup='麺' and date='2006/12/31' and price.fee>=750; %\enter%
+------------+--------------------------+--------+-----------+
| date       | food                     | amount | foodgroup |
+------------+--------------------------+--------+-----------+
| 2006/12/31 | 味噌ラーメン             |     11 | 麺        |
| 2006/12/31 | 五目ラーメン             |     21 | 麺        |
| 2006/12/31 | チャーシューメン         |     11 | 麺        |
| 2006/12/31 | 味噌チャーシュー         |      9 | 麺        |
| 2006/12/31 | 味噌ねぎラーメン         |      7 | 麺        |
+------------+--------------------------+--------+-----------+
5 rows in set (0.00 sec)
\end{lstlisting}
\end{enumerate}
\subsection{WordPressとか}
\end{document}