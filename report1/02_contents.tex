\chapter{メールサーバに関する内容}
\begin{wrapfigure}{r}[0mm]{.43\textwidth}
    \input{02_mail.tikz}
    \caption{メール送受信フロー}
    \label{fig:メール送受信フロー}
    \vspace{-1cm}
\end{wrapfigure}
\newcommand{\smtp}{\texttt{SMTP}}
\newcommand{\pop}{\texttt{POP}}
\newcommand{\imap}{\texttt{IMAP}}
\section{メール送受信フロー}
ここで,メールを送信してから宛先に届くまでの大まかな手続きを確認する(\figref{fig:メール送受信フロー}).
送り手は,MUA(Mail User Agent)と呼ばれるソフトウェア(ThunderbirdやMicrosoft Outlookなど)から\smtp を利用してMTAにメールを転送する.
この際,MUAはMTAのドメイン名を保持しており,DNSを用いてMTAのIPアドレスを取得後,\smtp を利用してMTAへメールを転送する.
転送されたメッセージは,\smtp を利用して宛先MTAへ転送され,Mail Spool\footnote{宛先のMUAが,メールを受け取るまで一時的に保存する場所.}(以下,Spool)へ保存される.
Spool内のメールは,\pop や\imap を利用して宛先のMUAへ転送されると,Spoolから削除される.
\section{aliasファイル}
通常,メールサーバを設定をしない限り,メールアドレスは「\texttt{username@domain}」になる.
ここで,ドメインはDNSサーバのMXレコードで指定されるドメイン名やAレコードで指定されるメールサーバのFQDNである.

\section{名前解決フロー}
DNSを利用して,ドメイン名に対してIPアドレスを割り出すことを「名前解決」と呼ぶ.