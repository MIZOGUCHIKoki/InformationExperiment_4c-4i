\section{内容}
\begin{wrapfigure}{r}[0mm]{.48\textwidth}
    \vspace{-.5cm}
    \begin{tikzpicture}
    \node (mua1){\includegraphics[keepaspectratio,width=0.1\textwidth]{../network_iconset/laptop.pdf}};
    \node at (mua1.north){\texttt{MUA}(送信)};
    \node[right=0.5cm of mua1,draw,text width=1cm,minimum height=1cm](mta1){};
    \node[fill=white] at (mta1.north){\texttt{MTA}};
    \node[right=0.5cm of mta1,draw,dotted,thick,text width=1cm, minimum height=1cm](spool1){};
    \node[fill=white] at (spool1.north){\texttt{MDA}};
    \node[right=0.5cm of spool1,draw,dashed,thick,text width=1cm, minimum height=1cm](popper1){};
    \node[fill=white] at (popper1.north){\texttt{\footnotesize Spool}};
    \node[below=1.5cm of popper1,draw,text width=1cm,minimum height=1cm](mta2){};
    \node[fill=white] at (mta2.south){\texttt{MTA}};
    \node[left=0.5cm of mta2,draw,dotted,thick,text width=1cm, minimum height=1cm](spool){};
    \node[fill=white] at (spool.south){\texttt{MDA}};
    \node[left=0.5cm of spool,draw,dashed,thick,text width=1cm, minimum height=1cm](popper){};
    \node[fill=white] at (popper.south){\texttt{\footnotesize Spool}};
    \node at (mua1 |- popper)(mua2){\includegraphics[keepaspectratio,width=0.1\textwidth]{../network_iconset/laptop.pdf}};
    \node at (mua2.south){\texttt{MUA}(受信)};
    \node[inner ysep=0.4cm,fit={(mta1)(spool1)(popper1)},draw,thick,rounded corners](wrap){};
    \node[fill=white] at (wrap.north){\texttt{\small Mail Server}};
    \node[inner ysep=0.3cm,fit={(mta2)(spool)(popper)},draw,thick,rounded corners](wrap2){};
    \node[fill=white] at (wrap2.south){\texttt{\small Mail Server}};
    \coordinate (A) at ($(spool1.center)!0.5!(spool.center)$);
    \draw[line width=1.5mm,draw=white](mua1.center)--(mta1.center)--(spool1.center)|-(A)-|(mta2.center)--(spool.center)--(popper.center)--(mua2.center);
    \draw[thick,-Stealth,dash dot](mua1.center)--(mta1.center);
    \draw[thick,-Stealth](mta1.center)--(spool1.center);
    \draw[thick,-Stealth,dash dot](spool1.center)|-(A)-|(mta2.center);
    \draw[thick,-Stealth](mta2.center)--(spool.center);
    \draw[thick,-Stealth](spool.center)--(popper.center);
    \draw[thick,-Stealth,densely dotted](popper.center)--(mua2.center);
    \coordinate (C) at ($(mua1.center)!0.5!(mta2.center)$);
    \draw[thick,-Stealth,dash dot] ($(C)+(-0.5,-2.7cm)$)--($(C)+(0.5,-2.7cm)$)node[right]{\texttt{SMTP}};
    \draw[thick,-Stealth,densely dotted] ($(C)+(-0.5,-3cm)$)--($(C)+(0.5,-3cm)$)node[right]{\texttt{POP}};
\end{tikzpicture}
    \caption{メッセージの流れ\footnotemark[1]}
    \label{fig:メール送受信フロー}
    \vspace{-.5cm}
\end{wrapfigure}
\footnotetext[1]{本レポートで用いる図の一部はヤマハ株式会社の公開済図形(\url{https://network.yamaha.com/support/download/tool/})を利用している.}
\newcommand{\smtp}{SMTP}
\newcommand{\pop}{POP}
\newcommand{\imap}{IMAP}
\paragraph{メール送受信フロー}
ここで,メールを送信してから宛先に届くまでの大まかな手続きを確認する(\figref{fig:メール送受信フロー}).
送り手は,MUA(Mail User Agent)と呼ばれるソフトウェア(ThunderbirdやMicrosoft Outlookなど)から\smtp を利用してMTAにメッセージを転送する.
この際,MUAはMTAのドメイン名を保持しており,DNSの\texttt{MX}レコードを用いてMTAのホスト名を取得後,\texttt{A}レコードを用いて送信先のIPアドレスを調べる.
MDA(Mail Delivery Agent)に転送されたメッセージは,取得したIPアドレス宛に\smtp で転送される.MDAはメッセージをMTAにより指定された送信先に転送すること役割を担う.
転送されたメッセージは,\smtp を利用して宛先MTAへ転送され,Mail Spool(以下,Spool)へ保存される.
Spoolとは,宛先のMUAが,メッセージを受け取るまで一時的に保存する場所である.
Spool内のメッセージは\pop を利用して宛先のMUAへ転送されると,Spoolから削除される.
\hfill\cite[p.9\ -\ p.11, p.13]{2004postfix詳解}
\paragraph{名前解決フロー}
ドメイン名に対して,対応付けられたIPアドレスへの名前解決をする.
名前解決をする主な方法としてDNS,ブロードキャストで問い合わせる方法\cite[p.135]{いちばんやさしいネットワークの本},WINS(Windows Internet Name Service),NIS(Network Information Service),\texttt{HOSTS}ファイルの適用などがある.
ここではOSに依存せず,データベースを分散管理でき,大規模ネットワークでも耐えうるDNSを用いた名前解決方法を適用する.

\begin{wrapfigure}{r}[0mm]{.45\textwidth}
    \centering
    \begin{tikzpicture}
    \newcommand{\server}{\includegraphics[keepaspectratio,width=0.1\textwidth]{../network_iconset/server.pdf}}
    \node(root){\server};
    \node[below=.1cm of root](jp){\server};
    \node[below=.1cm of jp](ac){\bfseries\Large\vdots};
    \node[below=.1cm of ac](info){\server};
    \coordinate (A) at ($(root)!0.5!(info)$);
    \node[left=2.8cm of A](ns){\server};
    \node[below=.5cm of ns](rs){\includegraphics[keepaspectratio,width=0.1\textwidth]{../network_iconset/laptop.pdf}};
    \node[below=-.5cm of rs]{\scriptsize リゾルバ};
    \foreach \u \v in {root/{\normalfont ``''}(root),jp/jp,info/info}{
    \node[right=-.3cm of \u]{\ttfamily\scriptsize\v};
    }
    \draw[thick,-latex]($(rs.north)+(-.1cm,0)$)--($(ns.south)+(-.1cm,0)$)node[midway,left]{\tiny\ttfamily 1.www Addr?};
    \draw[thick,-latex]($(ns.north)+(-.1cm,0)$)|-($(root.west)+(0,0.1cm)$)node[midway,above]{\tiny\ttfamily 2.jp's Addr?};
    \draw[thick,-latex]($(root.west)+(0,-0.1cm)$)-|($(ns.north)+(.1cm,0)$)node[midway,below right]{\tiny\ttfamily 3.jp's Addr};
    \draw[thick,-latex]($(ns.east)+(0,.2cm)$)--($(jp.west)+(0,.1cm)$)node[midway,above=-.1cm]{\rotatebox{12}{\tiny\ttfamily 4.ac's Addr?}};
    \draw[thick,-latex]($(jp.west)+(0,0cm)$)--($(ns.east)+(0,.1cm)$)node[midway,below=-.1cm]{\rotatebox{12}{\tiny\ttfamily 5.ac's Addr}};
    \draw[thick,-latex]($(ns.east)+(0,0.-.1cm)$)--($(info.west)+(0,.1cm)$)node[midway,above=-.3cm]{\rotatebox{-50}{\tiny\ttfamily 10.www's Addr?}};
    \draw[thick,-latex]($(info.west)+(0,-.05cm)$)--($(ns.east)+(0,-.25cm)$)node[midway,below=-.3cm]{\rotatebox{-50}{\tiny\ttfamily\hspace{1em} 11.www's Addr}};
    \draw[thick,-latex]($(ns.south)+(.1cm,0)$)--($(rs.north)+(.1cm,0)$)node[midway,right,align=left]{\tiny\ttfamily 12.www's Addr};
    \node[fill=white,opacity=.5,text=black,text opacity=1] at (ns.center){\scriptsize ネームサーバ};
\end{tikzpicture}
    \caption{名前解決}
    \vspace{-.5cm}
\end{wrapfigure}
DNSを用いた名前解決の手順を解説する.
ここでは例としてWebサイト\texttt{www.info.kochi-tech.ac.jp}のIPアドレスを,DNSを用いて取得する.
まず,リゾルバ(ネームサーバへアクセスするクライアント)は,自身のネットワーク設定により指定されたネームサーバへ\texttt{www.info.kochi-tech.ac.jp}のIPアドレスを問い合わせる.
そのネームサーバが,\texttt{www.info.kochi-tech.ac.jp}のIPアドレスを持っていない場合,まずは\texttt{root}ネームサーバへ,\texttt{jp}ネームサーバのIPアドレスを問い合わせる.
次に\texttt{jp}ネームサーバへ\texttt{ac}ネームサーバのIPアドレスを問い合わせる.これを繰り返し,最後に\texttt{info}ネームサーバが\texttt{www}のIPアドレスを,リゾルバが最初に問い合わせたネームサーバへ返し,
リゾルバへ応答する.\hfill\cite[p.8]{DNSBIND}
\newcommand{\aliases}{\texttt{aliases}}
\newcommand{\user}{ユーザ}
\paragraph{\texttt{\bfseries aliases} ファイル}
通常,メールサーバを設定をしない限り,メールアドレスは「\texttt{username@domain}」になる.
ここで,\texttt{username@domain}の\user 名\texttt{username}を公開したくない場合,
\aliases ファイルに\texttt{anonymous}と\texttt{username}を対応づける設定をすることで,
\texttt{anonymous@domain}でのメール受信が可能になる.
この機能は\user 名を秘匿にしたい場合だけでなく,たとえば\texttt{usernameA},\texttt{usernameB},\texttt{usernameC}の3\user を1つのエイリアス\texttt{group}と対応づけて,
\texttt{group@domain}宛に送信したメールを3\user で受信できる.