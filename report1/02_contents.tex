\chapter{内容}
\begin{wrapfigure}{r}[0mm]{.43\textwidth}
    \begin{tikzpicture}
    \node (mua1){\includegraphics[keepaspectratio,width=0.1\textwidth]{../network_iconset/laptop.pdf}};
    \node at (mua1.north){\texttt{MUA}(送信)};
    \node[right=0.5cm of mua1,draw,text width=1cm,minimum height=1cm](mta1){};
    \node[fill=white] at (mta1.north){\texttt{MTA}};
    \node[right=0.5cm of mta1,draw,dotted,thick,text width=1cm, minimum height=1cm](spool1){};
    \node[fill=white] at (spool1.north){\texttt{MDA}};
    \node[right=0.5cm of spool1,draw,dashed,thick,text width=1cm, minimum height=1cm](popper1){};
    \node[fill=white] at (popper1.north){\texttt{\footnotesize Spool}};
    \node[below=1.5cm of popper1,draw,text width=1cm,minimum height=1cm](mta2){};
    \node[fill=white] at (mta2.south){\texttt{MTA}};
    \node[left=0.5cm of mta2,draw,dotted,thick,text width=1cm, minimum height=1cm](spool){};
    \node[fill=white] at (spool.south){\texttt{MDA}};
    \node[left=0.5cm of spool,draw,dashed,thick,text width=1cm, minimum height=1cm](popper){};
    \node[fill=white] at (popper.south){\texttt{\footnotesize Spool}};
    \node at (mua1 |- popper)(mua2){\includegraphics[keepaspectratio,width=0.1\textwidth]{../network_iconset/laptop.pdf}};
    \node at (mua2.south){\texttt{MUA}(受信)};
    \node[inner ysep=0.4cm,fit={(mta1)(spool1)(popper1)},draw,thick,rounded corners](wrap){};
    \node[fill=white] at (wrap.north){\texttt{\small Mail Server}};
    \node[inner ysep=0.3cm,fit={(mta2)(spool)(popper)},draw,thick,rounded corners](wrap2){};
    \node[fill=white] at (wrap2.south){\texttt{\small Mail Server}};
    \coordinate (A) at ($(spool1.center)!0.5!(spool.center)$);
    \draw[line width=1.5mm,draw=white](mua1.center)--(mta1.center)--(spool1.center)|-(A)-|(mta2.center)--(spool.center)--(popper.center)--(mua2.center);
    \draw[thick,-Stealth,dash dot](mua1.center)--(mta1.center);
    \draw[thick,-Stealth](mta1.center)--(spool1.center);
    \draw[thick,-Stealth,dash dot](spool1.center)|-(A)-|(mta2.center);
    \draw[thick,-Stealth](mta2.center)--(spool.center);
    \draw[thick,-Stealth](spool.center)--(popper.center);
    \draw[thick,-Stealth,densely dotted](popper.center)--(mua2.center);
    \coordinate (C) at ($(mua1.center)!0.5!(mta2.center)$);
    \draw[thick,-Stealth,dash dot] ($(C)+(-0.5,-2.7cm)$)--($(C)+(0.5,-2.7cm)$)node[right]{\texttt{SMTP}};
    \draw[thick,-Stealth,densely dotted] ($(C)+(-0.5,-3cm)$)--($(C)+(0.5,-3cm)$)node[right]{\texttt{POP}};
\end{tikzpicture}
    \caption{メール送受信フロー\footnotemark[2]}
    \label{fig:メール送受信フロー}
    \vspace{-1cm}
\end{wrapfigure}
\newcommand{\smtp}{\texttt{SMTP}}
\newcommand{\pop}{\texttt{POP}}
\newcommand{\imap}{\texttt{IMAP}}
\section{メール送受信フロー}
ここで,メールを送信してから宛先に届くまでの大まかな手続きを確認する(\figref{fig:メール送受信フロー}).
送り手は,MUA(Mail User Agent)と呼ばれるソフトウェア(ThunderbirdやMicrosoft Outlookなど)から\smtp を利用してMTAにメールを転送する.
この際,MUAはMTAのドメイン名を保持しており,DNSを用いてMTAのIPアドレスを取得後,\smtp を利用してMTAへメールを転送する.
転送されたメッセージは,\smtp を利用して宛先MTAへ転送され,Mail Spool\footnote{宛先のMUAが,メールを受け取るまで一時的に保存する場所.}(以下,Spool)へ保存される.
Spool内のメールは,\pop を利用して宛先のMUAへ転送されると,Spoolから削除される.
\footnotetext[2]{本レポートで用いる図の一部はヤマハ株式会社の公開済図形(\url{https://network.yamaha.com/support/download/tool/})を利用している.}
\section{名前解決フロー}
DNSを利用して,ドメイン名とIPアドレスの対応づけ,その対応を呼び出して利用することを「名前解決」と呼ぶ.
ドメインに関しては\chapref{chap:DNS}で詳解する.
DNSにおけるデータは,リソースレコードと呼ばれる各種レコードで構成される.
IPアドレスやネームサーバ,ホスト名エイリアス,メールルーティングなどを記述することで,DNSを構成する\cite[p.88]{Postfix実用ガイド}.
以下には代表的なレコードと役割は以下の通りである.
\begin{oframed}
    \begin{description}
        \item[A] ホスト名とIPアドレスを対応づける.
        \item[CNAME] エイリアスホスト名が指し示す「正規名」を提供する.
            たとえば,ホスト名が\texttt{serevr}であるサーバをWebサーバとして利用したいとき,\texttt{www}をCNAMEを用いて\texttt{server}と対応づけると,\texttt{server}へのアクセスがホスト名を\texttt{www}に指定しても可能になる.
        \item[MX] メールルーティングを示す.MXレコードには,ドメインのメールエクスチェンジャを指定する.つまり,ドメイン宛のメールを処理するメールハブの名前を指定する.
            メールをどの場所に送れば良いかMTAに教える役割を果たしている.
    \end{description}
    \hfill\cite[p.88\ -\ p.89]{Postfix実用ガイド}
\end{oframed}
レコードの設定は,\chapref{chap:ゾーンの設定}で詳解する.
\clearpage
\newcommand{\alias}{\texttt{alias}}
\newcommand{\user}{ユーザ}
\section{\texttt{\bfseries alias} ファイル}
通常,メールサーバを設定をしない限り,メールアドレスは「\texttt{username@domain}」になる.
ここで,\texttt{username@domain}の名\texttt{username}宛に送りたいが,\user 名\texttt{username}を公開したくない場合,
\alias ファイルに\texttt{dummy}と\texttt{username}を対応づける設定をすることで,
\texttt{dummy@domain}でのメール受信が可能になる.
この機能は\user 名を秘匿にしたい場合だけでなく,たとえば\texttt{usernameA},\texttt{usernameB},\texttt{usernameC}の3\user を1つのエイリアス\texttt{group}と対応づけて,
\texttt{group@domain}宛に送信したメールを3\user で受信できる.
