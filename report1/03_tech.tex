\section{要素技術}
\subsection{ドメイン}\label{chap:domain}
\newcommand{\popt}{\texttt{POP3}}
\begin{wrapfigure}{r}[0mm]{.55\textwidth}
    \vspace{-1.5cm}
    \centering
    \newcommand{\tkin}[3]{\tikz[remember picture,baseline=(#1.base)]{\node[inner xsep=2mm,rounded corners,text height=3mm,text depth=1.5mm,draw](#1){#2}}}
{\Large\ttfamily \tkin{u1}{www}{red}.\tkin{u2}{info}{green}.\tkin{u3}{kochi-tech}{blue}.\tkin{u4}{ac}{yellow}.\tkin{u5}{jp}{gray}}
\begin{tikzpicture}[remember picture,overlay]
      \foreach \u \v in {u1/{ホスト名}, u2/{情報学群ラベル}, u3/{高知工科大学ラベル}, u4/{\texttt{ac}ラベル}, u5/{\texttt{jp}ラベル}}{
      \node[below=0.2cm of \u]{\scriptsize \v};
      }
\end{tikzpicture}
    \vspace{1em}
    \caption{高知工科大学情報学群ホームページURI}
    \vspace{-.5cm}
\end{wrapfigure}
DNSはUNIXファイルシステムと同様に木構造を持っている.ファイルシステムの根(root)は\texttt{\ /\ }であるのに対して,DNSの根(ルートDNS)は空ラベル(``'')で表現される.
UNIXファイルシステムにおける木構造の各節は,「ディレクトリ」であり,DNSにおいてはドメインである.
DNSでは木構造全体の,ルートまでのラベルをドット(``\texttt{.}'')で連結したものがDNSの絶対名となる.これをFQDN(Full Qualified Domain Name)と呼ぶ.
ホスト名\texttt{www.info.kochi-tech.ac.jp}のFQDNは\texttt{www.info.kochi-tech.ac.jp.}である.最後の``\texttt{.}''は,後ろに空ラベル(``'')があることを表現している.
当然,FQDNは世界に同じものが存在しないように作成する必要がある.
つまり,ドメインとは木構造の部分木を構成するものであり,部分木における根のFQDNがその部分木のドメイン名となる.
DNSでは,各ドメインをいくつかのサブドメインに分割し,サブドメインの管理をサブドメイン内で完結できる.
たとえば,ドメイン\texttt{info.kochi-tech.ac.jp}において,\texttt{jp}のサブドメインである\texttt{ac.jp}は,\texttt{ac.jp}で終わるドメインについて責任を持つ.
また,\texttt{ac.jp}は,\texttt{kochi-tech}というサブドメインをもち,\texttt{kochi-tech.ac.jp}が末尾につくドメイン名の責任を高知工科大学に委任している.

\begin{wrapfigure}{r}[0mm]{.62\textwidth}
    \centering
    \tikzset{point/.style={circle,minimum size=3pt,draw,fill=white}}
\begin{tikzpicture}
    \newcommand{\dist}{1.5cm}
    \node[point] (root){};\node[right=.2cm of root]{\texttt{root}};;
    \node[point,below left=\dist of root](jp){};\node[right=.2cm of jp]{\texttt{jp}};
    \node[point,below right=\dist of root](uk){};\node[right=.2cm of uk]{\texttt{uk}};
    \node[point,below left=\dist of jp](ac){};\node[right=.2cm of ac]{\texttt{ac}};
    \node[point,below right=\dist of jp](co){};\node[right=.2cm of co]{\texttt{co}};
    \node[point,below left=\dist of ac](kut){};\node[left=.2cm of kut](lkut){\texttt{kochi-tech}};
    \node[point,below right=\dist of ac](uec){};\node[right=.2cm of uec]{\texttt{uec}};
    \node[point,below left=\dist of kut](info){};\node[left=.2cm of info](linfo){\texttt{info}};
    \node[point,below right=\dist of kut](sceng){};\node[below=.2cm of sceng](lsceng){\texttt{sceng}};
    \node[point,fill=black,below left=\dist of info](www){};\node[below=.2cm of www](lwww){\texttt{www}};
    \node[point,fill=black,below right=\dist of info](host2){};\node[below=.2cm of host2](lhost2){\texttt{host2}};
    \node[point,fill=black]at($(www)!0.5!(host2)$)(host1){};\node[below=.2cm of host1](lhost1){\texttt{host1}};
    \draw[thick]{
        (root)--(jp)--(ac)--(kut)--(info)--(www)
        (kut)--(sceng)
        (info)--(host1)
        (info)--(host2)
        (root)--(uk)
        (jp)--(co)
        (ac)--(uec)
    };
    \node[ellipse,minimum width=3.5cm,minimum height=1.5cm,fill=gray,opacity=0.1]at (ac)(aczoned){};
    \node[above left=3cm of ac](aczone){\texttt{ac}ゾーン};
    \node[fit={(kut)(sceng)(info)(www)(host1)(host2)(lkut)(lsceng)(linfo)(lwww)(lhost1)(lhost2)},draw,rounded corners](warp_kut){};
    \node[right=1cm of warp_kut](lwarp_kut){高知工科大学が管理};
    \draw[-latex](aczone)to[bend right](aczoned.west);
    \draw[-latex](lwarp_kut)--(warp_kut.east);
    \node[inner sep=-.5mm,fit={(info)(www)(host1)(host2)(linfo)(lwww)(lhost1)(lhost2)},rounded corners,fill=gray,opacity=0.1](warp_infozone){};
    \node[below=.5cm of lwarp_kut](lwarp_infozone){\footnotesize\texttt{info.kochi-tech.ac.jp}ゾーン};
    \draw[-latex](lwarp_infozone.west)--(warp_infozone.east);
\end{tikzpicture}
    \caption{ドメインの木構造とゾーン}
    \label{fig:ドメインの木構造とゾーン}
    \vspace{-1cm}
\end{wrapfigure}
ここで,ゾーンという概念が登場する.ゾーンとは,名前空間において独立し,自治管理される部分である.
\figref{fig:ドメインの木構造とゾーン}での\texttt{info.kochi-tech.ac.jp}ゾーンには\texttt{www},\texttt{host1},\texttt{host2}の3クライアントがあり,これらは\texttt{info.kochi-tech.ac.jp}によりドメインを管理されている.
さらに,\texttt{ac.jp}ゾーンには\texttt{kochi-tech}ドメインは含まれていない.
これは「委任(委譲)」という行為で,\texttt{ac.jp}から,高知工科大学へ\texttt{kochi-tech}以下サブドメインの管理を委任する.
つまり,\texttt{ac.jp}は,\texttt{kochi-tech.ac.jp}のサブドメインを管理しないので,\texttt{ac.jp}のゾーンではない.
\hfill\cite[p.23]{DNSBIND}
\subsection{\bfseries\smtp と \popt}
\paragraph{\smtp}\smtp とは,OSI参照モデル第7層に属する,メール転送プロトコルである.ポート番号は\texttt{25}番である.
\figref{fig:メール送受信フロー}より,MUAからMTA,MTAからMTAへのメール転送に用いられる.これをメールリレーという.
\smtp サーバには\texttt{telnet}でアクセスし,コマンドを入力することでメールを配信する.
\texttt{telnet}で電子メールサーバにアクセスし,メールを配信する手続きを示す.ここでは\texttt{mail.example.com}へアクセスする.
\begin{lstlisting}[escapechar=\%,frame={single},caption={\smtp コマンド入力},label={src:smtpコマンド入力}]
$%\textbf{ telnet mail.example.com 25}%
Trying 10.232.45.151
Connected to mail.example.com.
Escape character is '^]'
220 mail.example.com ESMTP Postfix
%\textbf{HELLO mail.example.com}%
250 mail.oreilly.com
%\textbf{MAIL FROM: <info.oreilly.com>}%
250 Ok
%\textbf{RCPT TO: kdent@example.com}%
250 Ok

%\textbf{DATA}%
354 End data with <CR><LF>.<CR><LF>

%\textbf{Date: Mon, 8 Apr 2003 15:38:21 -0500}%
%\textbf{From: Customer Service <info@oreilly.com>}%
%\textbf{To: <kdant@example.com>}%
%\textbf{Reply-To: <service@oreilly.com>}%
%\textbf{Message-ID 01a4e2238200842@mail.oreilly.com}%
%\textbf{Subject: Have you read RFC 2822?}%

%\textbf{This is the start of the body of the message. It could continue}%
%\textbf{for many lines, but it dosen't.}%
.

250 Ok: queued as 5FA26B3DFE
%\textbf{quit}%
221 Bey
Connection closed by foreign host.
    \end{lstlisting}
\hfill 引用\cite[p.18]{Postfix実用ガイド}
また,\smtp には認証機能がないが,認証元が詐称されることを防ぐため,\texttt{POPbefreSMTP}や\smtp 認証が用いられる\cite[p.173]{インターネット工学}.
\paragraph{\popt}
\texttt{POP}とは,\figref{fig:メール送受信フロー}より,Spoolから外部MUAで受信するためのプロトコルである.
\smtp を用いたメッセージの受信は,送信されたメッセージが宛先MUAまで到達するため,宛先MUAが常にメール受信可能状態でなければならない.
これを解決するために\pop を用いることで,常に電源が入っている\pop サーバまで到達したメッセージはMUAの要求時にメッセージを受信できる.
\pop を用いてメッセージを転送した場合,Spool内のメールは削除される.\popt は\pop のバージョン3である.\hfill\cite[p.200, p.281]{マスタリングTCP/IP}
\subsection{第三者リレー}
オープンリレー,第三者中継とも呼ぶ.
オープンリレーは以下のように説明されている.
\begin{quote}
    ``メールリレーを誰にでも許可するメールサーバ''\hfill 引用\cite[p.53]{Postfix実用ガイド}
\end{quote}
つまり,オープンリレーを許可しているサーバでは,認証なく誰でもそのサーバを利用してメールを送信できるので,スパムメールなどの温床になる\cite[p.53]{Postfix実用ガイド}.