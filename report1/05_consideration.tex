\section{考察}
現在,DNSの抱えるリスクはセキュリティである.
DNSの安全性を高めるために,DNSSEC(DNS Security Extensions)が開発された.
DNSのセキュリティ脆弱性は,DNS応答レコードの偽造や改竄,正しいゾーン管理者により作られたゾーン情報でないことにある.
DNSSECは,応答レコードに公開鍵暗号方式を用いたディジタル署名を付加することで改竄検知できる.\par
しかしながら,DNSSECはDNS通信の傍受を防止することは目的としていない.リゾルバとDNSサーバ間の通信は暗号化されていない\cite{weko_75812_1}.
この通信を傍受することにより,リゾルバがどのサイトへアクセスしたか,通信を傍受している第三者が追跡できる.
これは,DNSサーバとリゾルバ間で公開鍵暗号方式を用いて鍵交換し,共通鍵暗号方式(\texttt{XOR})を用いて暗号化するハイブリッド暗号方式を施すことで対策できる.
しかし,DNSを用いた名前解決ごとに暗号化処理を施すのは,リゾルバ,DNSサーバともに負荷が大きくなるであろう.
簡易的な解決方法は,鍵交換の間隔を長くすることだが,大規模なネットワーク上のDNSでは,鍵管理に対して効率が悪くなる.
DNSの通信傍受に対する影響と,その暗号化については今後の研究テーマにしたい.