\section{作業記録}
\begin{table}[H]
    \begin{tabularx}{\textwidth}{clR}
        \hline
        \multirow{3}{*}{\bfseries サーバ}    & IPアドレス  & \texttt{192.168.0.161}                        \\
                                          & FQDN    & \texttt{server.c5.exp.info.kochi-tech.ac.jp.} \\
                                          & ホスト名の別名 & \texttt{www}                                  \\
        \hline
        \multirow{2}{*}{\bfseries クライアント} & IPアドレス  & \texttt{192.168.0.162}                        \\
                                          & FQDN    & \texttt{client.c5.exp.info.kochi-tech.ac.jp.} \\
        \hline
    \end{tabularx}
\end{table}
\subsection{DNSサーバ構築}
DNSサーバを構築する.今回はBINDというソフトウェアを用いてDNSを構築する.
\newcommand{\enter}{\ovalbox{Enter⏎}}
\begin{enumerate}
    \item BINDをインストールする.
          \begin{lstlisting}
$ sudo su %\enter%
# apt update %\enter%
# apt install bind9 %\enter%
    \end{lstlisting}
    \item BIND全体の設定として\texttt{named.conf}を操作する.
          \begin{lstlisting}[caption={\ttfamily /etc/bind/named.conf},style=file]
// This is the primary configuration file for the BIND DNS server named.
//
// Please read /usr/share/doc/bind9/README.Debian.gz for information on the
// structure of BIND configuration files in Debian, *BEFORE* you customize
// this configuration file.
//
// If you are just adding zones, please do that in /etc/bind/named.conf.local
include "/etc/bind/named.conf.options";
include "/etc/bind/named.conf.local";
include "/etc/bind/named.conf.default-zones";
\end{lstlisting}
    \item これから編集するファイル\texttt{named.conf.options}のデフォルトファイルを別名で保存する.
          \begin{lstlisting}
# cd /etc/bind %\enter%
# cp -p named.conf.options named.conf.options.org %\enter%
\end{lstlisting}
    \item ここではBINDが動作するワーキングディレクトリの指定をするために\texttt{named.conf.options}を編集する.
          \begin{lstlisting}[caption={\ttfamily /etc/bind/named.conf.options},style=file]
options {
    directory "/var/cache/bind";
};
\end{lstlisting}
    \item ゾーンファイルを指定する.\texttt{named.conf.local}にはゾーンファイルのパスを記述する.
          ドメイン名を\texttt{c5.exp.info.kochi-tech.ac.jp}に設定し,ゾーンファイルのパスを\texttt{/etc/bind/c5.zone}にする.
          \begin{lstlisting}[style=file,caption={\ttfamily /etc/bind/named.conf.local}]
zone "c5.exp.info.kochi-tech.ac.jp" {
    type primary;
    file "/etc/bind/c5.zone";
};
    \end{lstlisting}
    \item ルートゾーンの設定を確認する.ルートゾーンは``\texttt{.}''で表されており,ルートサーバのアドレスは\texttt{/usr/share/dns/root.hints}に記述されている.
          \begin{lstlisting}[caption={\ttfamily /etc/bind/named.conf.default-zones},style=file]
zone "." {
    type    hint;
    files   "/usr/share/dns/root.hints";
}
    \end{lstlisting}
    \item ゾーンを設定する.
          \begin{description}
              \item[\texttt{TTL}] キャッシュの有効期限を記す.
              \item[\texttt{SOA}レコード] ``このサーバがこのゾーンのデータとして最も信頼情報源であることを示すもの.''
          \end{description}
          \begin{lstlisting}[style=file,caption={\ttfamily /etc/bind/c5.zone}]
$TTL 100
@ IN SOA server.c5.exp.info.kochi-tech.ac.jp. postmaster.c5.exp.info.kochi-tech.ac.jp. (
    2023061901
    100
    100
    100
    100 )
IN          NS          server.c5.exp.info.kochi-tech.ac.jp.
server      IN  A       192.168.0.161
www         IN  CNAME   server
client      IN  A       192.168.0.162
    \end{lstlisting}

\end{enumerate}
\subsection{メールサーバの構築}