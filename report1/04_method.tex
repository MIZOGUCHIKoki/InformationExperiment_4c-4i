\section{作業記録}
\begin{table}[H]
    \caption{設定事項}
    \begin{tabularx}{\textwidth}{clR}
        \hline
        \multirow{5}{*}{\bfseries サーバ}    & IPアドレス                   & \texttt{192.168.0.161}                        \\
        \cline{2-3}
                                          & FQDN                     & \texttt{server.c5.exp.info.kochi-tech.ac.jp.} \\
        \cline{2-3}
                                          & \multirow{3}{*}{ホスト名の別名} & \texttt{www.c5.exp.info.kochi-tech.ac.jp.}    \\
                                          &                          & \texttt{smtp.c5.exp.info.kochi-tech.ac.jp.}   \\
                                          &                          & \texttt{pop.c5.exp.info.kochi-tech.ac.jp.}    \\
        \hline
        \multirow{2}{*}{\bfseries クライアント} & IPアドレス                   & \texttt{192.168.0.162}                        \\
        \cline{2-3}
                                          & FQDN                     & \texttt{client.c5.exp.info.kochi-tech.ac.jp.} \\
        \hline
    \end{tabularx}
\end{table}
\subsection{DNSサーバ構築}
BINDというソフトウェアを用いてDNSサーバを構築する.BINDのバージョンは\texttt{9.18.16}である.
\newcommand{\enter}{\ovalbox{Enter⏎}}
\begin{enumerate}
    \renewcommand{\labelenumi}{\textbf{\theenumi}.\ }
    \item パッケージマネジャ\texttt{apt}をアップデートした後,\texttt{apt}を用いてBINDをインストールする.
          \begin{lstlisting}
$ sudo su %\enter%
# apt update %\enter%
# apt install bind9 %\enter%
    \end{lstlisting}
    \item BIND全体の設定として\texttt{named.conf}を操作する.
          \begin{lstlisting}[caption={\ttfamily /etc/bind/named.conf},style=file]
// This is the primary configuration file for the BIND DNS server named.
//
// Please read /usr/share/doc/bind9/README.Debian.gz for information on the
// structure of BIND configuration files in Debian, *BEFORE* you customize
// this configuration file.
//
// If you are just adding zones, please do that in /etc/bind/named.conf.local
include "/etc/bind/named.conf.options";
include "/etc/bind/named.conf.local";
include "/etc/bind/named.conf.default-zones";
\end{lstlisting}
    \item これから編集するファイル\texttt{named.conf.options}のデフォルトファイルを別名で保存する.
          \begin{lstlisting}
# cd /etc/bind %\enter%
# cp -p named.conf.options named.conf.options.org %\enter%
\end{lstlisting}
    \item ここではBINDが動作するワーキングディレクトリの指定をするために\texttt{named.conf.options}を編集する.
          \begin{lstlisting}[caption={\ttfamily /etc/bind/named.conf.options},style=file]
options {
    directory "/var/cache/bind";
};
\end{lstlisting}
    \item ゾーンファイルを指定する.\texttt{named.conf.local}にはゾーンファイルのパスを記述する.
          ドメイン名を\texttt{c5.exp.info.kochi-tech.ac.jp}に設定し,ゾーンファイルのパスを\texttt{/etc/bind/c5.zone}にする.
          \begin{lstlisting}[style=file,caption={\ttfamily /etc/bind/named.conf.local}]
zone "c5.exp.info.kochi-tech.ac.jp" {
    type primary;
    file "/etc/bind/c5.zone";
};
    \end{lstlisting}
    \item ルートゾーンの設定を確認する.ルートゾーンは``\texttt{.}''で表されており,ルートサーバのアドレスは\texttt{/usr/share/dns/root.hints}に記述されている.
          \begin{lstlisting}[caption={\ttfamily /etc/bind/named.conf.default-zones},style=file]
zone "." {
    type    hint;
    files   "/usr/share/dns/root.hints";
}
    \end{lstlisting}
    \item ゾーンを設定する.最後に``\texttt{.}''がない場合はゾーンのドメイン名が補完される.
          \begin{description}
              \item[\bfseries\texttt{TTL}] キャッシュの有効期限を記す.
              \item[\bfseries\texttt{SOA}レコード] このサーバがゾーン内で最も信頼するに値するものであることを示すもの.\texttt{SOA}レコードの中には次の項目を設定する.
                  \begin{itemize}
                      \item サーバのFQDN,メールアドレスのFQDN(``\texttt{@}''を``\texttt{.}''に置換する).
                      \item シリアル番号.
                      \item セカンダリサーバがプライマリサーバに情報更新がないか問い合わせる間隔.
                      \item リフレッシュ失敗時に,際リフレッシュする時間間隔.
                      \item 生存時間.
                  \end{itemize}
              \item[\bfseries\texttt{NS}レコード] ネームサーバのFQDNを記す.
              \item[\bfseries\texttt{A}レコード]  ホスト名とIPアドレスを対応づける.
              \item[\bfseries\texttt{CNAME}レコード] ホスト名の別名を設定する.
              \item[\bfseries\texttt{MX}レコード] メールアドレスのドメイン名とサーバのFQDNを対応づける.\hfill\cite[p.69\ -\ p.71]{DNSBIND}
          \end{description}
          \begin{lstlisting}[style=file,caption={\ttfamily /etc/bind/c5.zone}]
$TTL 100
@ IN SOA server.c5.exp.info.kochi-tech.ac.jp. postmaster.c5.exp.info.kochi-tech.ac.jp. (
    2023061901
    100
    100
    100
    100 )
            IN  NS      server.c5.exp.info.kochi-tech.ac.jp.
server      IN  A       192.168.0.161
client      IN  A       192.168.0.162
www         IN  CNAME   server
smtp        IN  CNAME   server
pop         IN  CNAME   server

c5.exp.info.kochi-tech.ac.jp IN MX server.c5.exp.info.kochi-tech.ac.jp.
    \end{lstlisting}
    \item
\end{enumerate}
\subsection{メールサーバの構築}