\section{目的}
本実験の目的は,電子メールサーバ(以下,メールサーバ)の構築である.
\subsection{構築する理由}
電子メール(以下,メール)は,言わばコンピュータネットワーク上の「郵便」のようなものである.
メールの誕生により,文字や写真などバイナリファイルの共有を,ネットワークを通して行えるようになった.
つまり,非接触での情報伝達が電話からインターネットへと変革するきっかけとなったのである.
本実験では,メールサーバを作成し,同一ネットワーク上のコンピュータ間でメールのやりとりを行えるようにする.
さらに,メールサーバ構築に際して,ドメインの設定をするために,名前解決をするシステムを構築する.
\subsection{構築するものの概要}
メールを支える技術は多数あるが,本実験ではメールサービスの根幹を成す,MTA(Mail Transfer Agent)を構築し,1つの宛先アドレスに対して複数アドレスに送信できるメーリングリスト機能を持たせる.
ドメインの設定もメールサーバ機能を有するサーバで完結できるように,同サーバ内にDNSも構築する.
