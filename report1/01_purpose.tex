\section{目的}
本実験の目的は,電子メールサーバ(以下,メールサーバ)の構築である.
メールサーバ構築に際して,ドメインの設定が必要になる.
ドメインの設定もメールサーバ機能を有するサーバで完結できるように,同サーバ内にDNSも構築する.
\subsection{電子メール}
電子メール(以下,メール)は,言わばコンピュータネットワーク上の「郵便」のようなものである.
メールの誕生により,文字や写真などバイナリファイルの共有を,ネットワークを通して行えるようになった.
つまり,非接触での情報伝達が電話からインターネットへと変革するきっかけとなったのである.
メールを支える技術は多数あるが,本実験ではメールサービスの根幹を成す,MTA(Mail Transfer Agent)を構築し,1つの宛先アドレスに対して複数アドレスに送信できるメーリングリスト機能を持たせる.
MTAについては\chapref{chap:smtppop}で詳解する.
\subsection{DNSの存在意義}
DNSが誕生する以前,ARPAnet\footnote{米国の重要な機関を結ぶ広域コンピュータネットワーク\cite[p.1]{DNSBIND}.}は数百台のホストだけで構成される小さなネットワークであり,
ARPAnet上すべてのホスト情報は,\texttt{HOSTS.TXT}というファイルで管理されていた.
\texttt{HOSTS.TXT}の更新は,SRI-NIC\footnote{SIRのネットワークインフォメーションセンター(NIC)のホスト.SIRは元のStanford Research Institute.\texttt{HOSTS.TXT}はSIRのNICによって管理されていた.}にFTP接続して行われており,
ARPAnetの管理者は,ホスト情報に変更があるとSRI-NICへ電子メールを用いて通知していた.
しかし,ARPAnetの規模が大きくなると,当然\texttt{HOSTS.TXT}のサイズが大きくなり,SRI-NICから更新情報を得ようとするホスト数も増加する\cite[p.3]{DNSBIND}.
ARPAnetがTCP/IPプロトコルに移行すると,ネットワーク上のホスト数が爆発的に増え,\texttt{HOSTS.TXT}での管理が困難になった.その主な原因は3つある.
\begin{enumerate}
    \item \textbf{トラフィックと負荷的問題}\\
          \texttt{HOSTS.TXT}配布に伴うプロセッサの負荷と,ネットワークトラフィックの増大にSRI-NICが耐えきれなかった.
    \item \textbf{名前の衝突問題}\\
          \texttt{HOSTS.TXT}は,同一ホスト名を登録できないが,ホスト名割り当てに権限がないNICはそれを防げなかった.
    \item \textbf{一貫性担保の問題}\\
          \texttt{HOSTS.TXT}の一貫性担保は困難であった.ARPAnetの管理者が新しい\texttt{HOSTS.TXT}に更新するころには,ホストのアドレスが変わっていたり,新しいホストが追加されたりした.
\end{enumerate}
\hfill\cite[p.4]{DNSBIND}\par
ここで登場するのが,DNSである.DNSは分散データベースであり,\texttt{HOSTS.TXT}での管理とは異なり,データベースの各部分はローカルに管理されながら,それぞれの部分データはネットワーク全体で利用できる\cite[p.5]{DNSBIND}.
メールサーバ構築にあたって,DNSを提供するDNSサーバを構築し,ホスト名を用いた通信を実現する.