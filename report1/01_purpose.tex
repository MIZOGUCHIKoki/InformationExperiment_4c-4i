\section{目的}
本実験の目的は,電子メールサーバ(以下,メールサーバ)の構築である.
\subsection{構築する理由}
電子メール(以下,メール)は,言わばコンピュータネットワーク上の「郵便」のようなものである.
メールの誕生により,文字や写真などバイナリファイルの共有を,ネットワークを通して行えるようになった.
つまり,非接触での情報伝達が電話からインターネットへと変革するきっかけとなったのである.
本実験では,メールサーバを作成し,同一ネットワーク上のコンピュータ間でメールのやりとりを行えるようにする.
昨今,メールサービスは,ECなどのインターネットを用いたサービスのさまざまな通達に用いられる.そこで効率的にメールを配信するために,
1つのメールアドレスで複数宛に送信できるような機能がある.本実験ではそのような機能も併せて実現する.

\subsection{構築するものの概要}
メールを支える技術は多数あるが,本実験ではメールサービスの根幹を成す,MTA(Mail Transfer Agent)を構築し,1つの宛先アドレスに対して複数アドレスに配信できるメーリングリスト機能を持たせる.
また,メールの配送にはDNS(Domain Name System)が必要であるので,本実験ではDNSサーバも併せて構築する.