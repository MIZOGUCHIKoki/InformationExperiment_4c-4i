\section{考察}
昨今,官民問わずサイバー攻撃が後を絶たない.
近年は,組織的な犯罪で,金銭を目的とした商業的な,巧妙で対策の難しいサイバー攻撃が増加している\cite{動的ネットワーク構成によるサイバー攻撃対策支援手法の研究}.
無論,大学というインターネットに比べると小さな規模のネットワークでも,危険は潜んでいる.
たとえば,タグVLANは,EthernetヘッダにVLAN IDを付することで所属するセグメントを認識している.
パケットは簡単に書き換えられるので,悪意のあるものがVLAN IDを書き換え,所属するLANとは別のLANへパケットが漏れる可能性もある.
これは,A研究室とB研究室のネットワークをVLANで区切っている状況下では,A研究室のホストがB研究室のLANへアクセスできることを示唆している.
この対策として,VLAN IDをハッシュ化してヘッダに付し,さらに,スイッチ間でVLAN IDを同期し,VLAN IDをフレームごとに変更するワンタイムVLAN IDを導入することで解決できるのではないだろうか.