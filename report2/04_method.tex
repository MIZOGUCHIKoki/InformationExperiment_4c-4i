\section{作業記録}
\subsection{スイッチの操作}
\begin{enumerate}
    \renewcommand{\labelenumi}{\textbf{\theenumi.}\ }
    \item スイッチの電源を切り,スイッチからすべての配線を取り外す.コンソールケーブルをコンピュータのUSB端子に挿入し,\texttt{/dev/ttyUSB0}が存在するか確認したのち,以下のコマンドを実行する.
          \begin{lstlisting}
# chmod 777 /dev/ttyUSB0 
    \end{lstlisting}
    \item \texttt{cu}をインストールする.
          \begin{lstlisting}
# apt install cu            
        \end{lstlisting}
    \item コンソールに接続するため,以下のコマンドを実行する.
          \begin{lstlisting}
# cu -l ttyUSB0 -s 9600                
            \end{lstlisting}
    \item コンソールに接続できたら,スイッチのホスト名設定,パスワード設定,パスワードの暗号化,DNSルックアップの無効化,コンソールの設定をする
          \begin{lstlisting}
Switch> enable
Switch# configure terminal
Switch(config)# hostname switch5
switch5(config)# username exp password 0 root00
switch5(config)# service password-encryption
switch5(config)# no ip domain-lookup
switch5(config)# line console 0
switch5(config-line)# logging synchronous
\end{lstlisting}
    \item VLAN1へ,スイッチに遠隔ログイン用IPアドレスを割り当てる.
          \begin{lstlisting}
switch5(config)# interface vlan1
switch5(config-if)# ip address 192.168.0.58 255.255.255.0
switch5(config-if)# no shutdown
switch5(config-if)# exit
\end{lstlisting}
    \item \texttt{telnet}の許可設定をする.
          \begin{lstlisting}
switch5(config)# line vty 0 4
switch5(config-line)# password 0 root00
switch5(config-line)# login
switch5(config-line)# exit
\end{lstlisting}
    \item スパニングツリーを無効にする.
          \begin{lstlisting}
switch5(config)# no spanning-tree vlan 1
    \end{lstlisting}
    \item VLANを定義する.本実験では,1〜8番ポートをバックボーン用,9〜16番ポートを5C班用,17〜24番ポートを5i班用のVLANと定義する.
          また,VLAN番号は班サブネットの第3オクテットとする.
          \begin{lstlisting}
switch5(config)# vlan 26
switch5(config-vlan)# name group5c
switch5(config-vlan)# exit
switch5(config)# vlan 15
switch5(config-vlan)# name group5i
switch5(config-vlan)# exit
switch5(config)# interface range fa0/9-16
switch5(config-if-range)# switchport mode access
switch5(config-if-range)# switchport access vlan 26
switch5(config-if-range)# exit
switch5(config)# interface range fa0/17-24
switch5(config-if-range)# switchport mode access
switch5(config-if-range)# switchport access vlan 15
switch5(config-if-range)# exit
\end{lstlisting}
\end{enumerate}
\subsection{配線}
\figref{fig:ルータとスイッチの配線}のように,ルータとスイッチ間を接続し,サーバとクライアントをスイッチと接続する.
\begin{figure}
    \centering
    \begin{tikzpicture}
        \tikzset{port/.style={rectangle,minimum width=1cm,minimum height=1cm,draw}};
        \node[port,fill=yellow!70!black!30](p00){};
        \node[port,below=.2cm of p00,fill=yellow!70!black!30](p10){};
        \foreach \u[evaluate=\u as \v using \u-1] in {1,2,...,11}{
                \node[fill=\ifnum\u<4yellow!70!black!30\else\ifnum\u<8magenta!30\else cyan!30\fi\fi ,port,right=.2cm of p0\v](p0\u){};
                \node[fill=\ifnum\u<4yellow!70!black!30\else\ifnum\u<8magenta!30\else cyan!30\fi\fi ,port,right=.2cm of p1\v](p1\u){};
            }
        \node[fit={(p00)(p111)},thick,draw](switch){};
        \node[fit={(p00)(p13)},inner sep=.3mm](backborn){};
        \node[fit={(p04)(p17)},inner sep=.3mm](5cvlan){};
        \node[fit={(p08)(p111)},inner sep=.3mm](5ivlan){};
        \node[below=.3cm]at(backborn.south){backborn用VLAN};
        \node[below=.3cm]at(5cvlan.south){5C班用VLAN};
        \node[below=.3cm]at(5ivlan.south){5i班用VLAN};
        \node[above=1cm of p05,port,fill=gray!50](rp1){};
        \node[above=1cm of p08,port,fill=gray!50](rp2){};
        \node[above=.1cm of rp2,port,fill=gray!50](rp3){};
        \node[above=1cm of p00,rectangle,minimum width=1cm,minimum height=1cm](nd1){};
        \node[above=1cm of p011,rectangle,minimum width=1cm,minimum height=1cm](nd2){};
        \node[above]at(rp1.north){\ttfamily FE2};
        \node[right]at(rp2.east){\ttfamily FE0};
        \node[right]at(rp3.east){\ttfamily FE1};
        \node[fit={(nd1)(nd2)(rp3)},draw,thick](ruter){};
        \node[left]at(switch.west){\rotatebox{90}{Switch}};
        \node[left]at(ruter.west){\rotatebox{90}{Ruter}};
        \draw[ultra thick]{
            ($(p10.center)+(-1cm,-1cm)$)--($(p10.center)+(0,-1cm)$)node[midway,below]{\tiny backborn}--(p10.center)
            (p03.center)|-(rp1.center)
            (p06.center)|-(rp3.center)
            (p08.center)--(rp2.center)
            (p14.center)|-($(p14.center)+(-1cm,-1cm)$)node[midway,above left]{\tiny Serverへ}
            (p17.center)|-($(p17.center)+(1cm,-1cm)$)node[midway,above right]{\tiny Clientへ}
        };
        \foreach \u in {p03,p10,p06,p08,rp1,rp2,rp3,p14,p17}{
                \node[circle,fill=black,radius=.3cm]at(\u){};
            }
        \foreach \u[evaluate=\u as \v using \u*2+1,evaluate=\v as \w using \v+1] in {0,1,...,11}{
                \node[below right,minimum size=.3cm,anchor=north west]at(p0\u.north west){\bfseries\tiny\pgfmathprintnumber[fixed,fixed zerofill,precision=0]{\v}};
                \node[below right,minimum size=.3cm,anchor=north west]at(p1\u.north west){\bfseries\tiny\pgfmathprintnumber[fixed,fixed zerofill,precision=0]{\w}};
            }
        \begin{scope}[on background layer]
            \fill[gray!20](switch.north east)rectangle (switch.south west);
            \fill[gray!20](ruter.north east)rectangle (ruter.south west);
        \end{scope}
    \end{tikzpicture}
    \caption{ルータとスイッチの配線}
    \label{fig:ルータとスイッチの配線}
\end{figure}