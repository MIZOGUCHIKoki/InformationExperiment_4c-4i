\section{要素技術}
\subsection{ルーティングプロトコル}
AS(Autonomous System)とは,経路制御に関するルールを決めて,それをもとに運用する範囲を指す.
AS内部の経路制御ではIGP(Interior Gateway Protocol),AS間の経路制御ではEGP(Exterior Gateway Protocol)を用いる\cite[p.173]{マスタリングTCPIP}.
\subsubsection{経路制御アルゴリズム}
経路制御のアルゴリズムは大きく2つある.
\paragraph{距離ベクトル型}
距離と方向によってネットワークやホストの位置を決定し,これらの情報から経路制御表を作成する.
処理は比較的簡単だが,ルータ間で交換される情報は距離と向きだけなので,ネットワークが複雑になると,経路の収束\footnote{経路制御情報が安定すること.}に時間がかかる\cite[p.174\ -\ p.175]{マスタリングTCPIP}.
\paragraph{リンク状態型}
ルータがネットワーク全体の接続状態を理解して経路制御表を作成する方法.
ネットワークの構造は,どのルータにとっても同じなので,すべてのルータが同じ経路制御情報を持つ.
ルータ間の経路制御情報をすばやく同期させれば,経路制御を安定させられる.
リンク状態型は複雑なネットワークでも安定した経路制御をできるが,欠点としてネットワークトポロジーから経路制御表を作成する計算コストが高い\cite[p.175]{マスタリングTCPIP}.
\subsubsection{主なルーティングプロトコル}
主なルーティングテーブルと,方式について説明する.ここでは,RIPとOSPFを取り上げる.
\begin{center}
    \begin{tabularx}{\textwidth}{c|C|C|C}
        \hline
        分類     & \multicolumn{2}{c|}{IGP} & EGP              \\
        \hline
        プロトコル名 & RIP                      & OSPF   & BGP     \\
        \hline
        アルゴリズム & 距離ベクトル型                  & リンク状態型 & 距離ベクトル型 \\
        \hline
    \end{tabularx}
\end{center}
\paragraph{RIP}RIP(Routing Information Protocol)は,距離ベクトル型のルーティングプロトコルである.
経路制御情報を30秒周期でブロードキャストし,情報を伝搬する(\figref{fig:RIPの経路情報交換}).
距離が一番短い,つまりホップ数が最小になる経路を選択する\cite[p.276\ -\ p.277]{マスタリングTCPIP}\cite[p.132]{インターネット工学}.
\begin{figure}
    \centering
    \begin{tikzpicture}
        \tikzset{dist/.style={midway,fill=red!10,text=black,align=center,text width=2cm,rounded corners}};
        \coordinate (ntA) at (0,0);
        \node[right=2.4cm of ntA](rt1){\includegraphics[keepaspectratio,width=2cm]{../network_iconset/router_01_nb.pdf}};
        \node[right=2.5cm of rt1](rt2){\includegraphics[keepaspectratio,width=2cm]{../network_iconset/router_01_nb.pdf}};
        \node[above=2cm of rt2](rt3){\includegraphics[keepaspectratio,width=2cm]{../network_iconset/router_01_nb.pdf}};
        \node[right=2.5cm of rt2](rt4){\includegraphics[keepaspectratio,width=2cm]{../network_iconset/router_01_nb.pdf}};
        \draw[ultra thick]{
            (ntA)--(rt1)node[midway,above]{Network1 (NW1)}
            (rt1)--(rt2)
            (rt2)--(rt3)
            (rt2)--(rt4)
            (rt3)-|(rt4)
            (rt3)--($(rt3.west)+(-1cm,0cm)$)
        };
        \foreach \u in{1,2,...,4}{
                \node[below=.2cm,text=white]at(rt\u){ルータ\u};
            }
        \draw[dashed,red!80!black,thick,-latex]($(rt1.east)+(0,-.2cm)$)--($(rt2.west)+(0,-.2cm)$)node[dist,below=.2cm]{\scriptsize ルータ1はNW1まで距離1};
        \draw[dashed,red!80!black,thick,-latex]($(rt1.west)+(0,-.2cm)$)--(0,-.2cm);
        \draw[dashed,red!80!black,thick,-latex]($(rt2.north)+(-.2cm,0cm)$)--($(rt3.south)+(-.2cm,0cm)$)node[dist,left=.2cm]{\scriptsize ルータ2はNW1まで距離2};
        \draw[dashed,red!80!black,thick,-latex]($(rt3.east)+(0cm,-.2cm)$)-|($(rt4.north)+(-.2cm,0cm)$)node[dist,below left=.2cm]{\scriptsize ルータ3はNW1まで距離3};
        \draw[dashed,red!80!black,thick,-latex]($(rt2.east)+(0,-.2cm)$)--($(rt4.west)+(0,-.2cm)$)node[dist,below=.2cm]{\scriptsize ルータ2はNW1まで距離2};
        \draw[dashed,red!80!black,thick,-latex]($(rt4.north)+(.2cm,0cm)$)|-($(rt3.east)+(0,.2cm)$)node[dist,right=.2cm]{\scriptsize ルータ4はNW1まで距離3};
    \end{tikzpicture}
    \caption{RIPの経路情報交換}
    \label{fig:RIPの経路情報交換}
\end{figure}
\paragraph{OSPF}
OSPF(Open Shortest Path First)は,リンク状態型のルーティングプロトコルで各リンクに重みをつけ,この重みが最小となるような経路を選択する.
ホップ数が最小でなくても,コストが最小の経路を選択する.
\figref{fig:OSPFでの経路選択}の例では,ホップ数が最小の経路はInternet VPNでのコストが大きいため,コストが最小となる経路を経路を選択している.
OSPFのコスト算出は,最大通信帯域が大きいほど小さな値が設定される\cite[p.138]{インターネット工学}.
\begin{figure}
    \centering
    \begin{tikzpicture}
        \node(hostA){\includegraphics[keepaspectratio,width=2cm]{../network_iconset/laptop_nb.pdf}};
        \node[right=1cm of hostA](rtA){\includegraphics[keepaspectratio,width=2cm]{../network_iconset/router_01_nb.pdf}};
        \node[below=1cm of rtA](rtB){\includegraphics[keepaspectratio,width=2cm]{../network_iconset/router_01_nb.pdf}};
        \node[right=4cm of rtB](rtE){\includegraphics[keepaspectratio,width=2cm]{../network_iconset/router_01_nb.pdf}};
        \node[above=1cm of rtE](rtD){\includegraphics[keepaspectratio,width=2cm]{../network_iconset/router_01_nb.pdf}};
        \node[right=1cm of rtD](hostB){\includegraphics[keepaspectratio,width=2cm]{../network_iconset/laptop_nb.pdf}};
        \node[anchor=south](rtC) at($(rtA)!0.5!(rtD)+(0,1.5cm)$){\includegraphics[keepaspectratio,width=2cm]{../network_iconset/router_01_nb.pdf}};
        \coordinate (a) at ($(rtA)!0.5!(rtB)$);
        \coordinate (b) at ($(rtD)!0.5!(rtE)$);
        \draw[ultra thick]{
            (hostA)--(hostA|-a)
            ($(hostA|-a)+(-.5cm,0)$)--($(a)+(.5cm,0)$)node[midway,align=center](la1){}
            (hostB)--(hostB|-b)
            ($(hostB|-b)+(.5cm,0)$)--($(b)+(-.5cm,0)$)node[midway,align=center](la2){}
            (rtA)--(a)
            (rtB)--(a)
            (rtD)--(b)
            (rtE)--(b)
        };
        \draw[ultra thick,densely dashed]{
            (rtA)|-(rtC)node[rounded corners,text opacity=1,fill opacity=.5,text width=3cm,fill=cyan!30,midway,align=center,left=-.4cm]{\scriptsize 広域インターネット網\\100Mbps\\コスト\(=100\)}
            (rtC)-|(rtD)node[rounded corners,text opacity=1,fill opacity=.5,text width=3cm,fill=cyan!30,midway,align=center,right=-.4cm]{\scriptsize IP-VPN網\\200Mbps\\コスト\(=90\)}
        };
        \draw[ultra thick,dash dot]{
            (rtB)--(rtE)node[rounded corners,text width=3cm,text opacity=1,fill opacity=.5,above=.1cm,fill=magenta!30,midway,align=center]{\scriptsize Internet VPN\\50Mbps\\コスト\(=600\)}
        };
        \foreach \u in{1,2}{
                \node[above] at (la\u){Ehternet 100Mbps};
                \node[below] at (la\u){コスト\(=100\)};
            }
        \draw[line width=.1cm,red,-latex](hostA)|-($(rtC.north)+(0,.5cm)$)-|(hostB);
        \node[above=.1cm] at($(rtC.north)+(0,.5cm)$){OSPFでの経路 【ホップ数\(=3\),総コスト\(=390\)】};
        \draw[line width=.1cm,gray,-latex]($(hostA|-a)+(0,-.5cm)$)|-($(rtB)!0.5!(rtE)+(0,-1.5cm)$)-|($(hostB|-a)+(0,-.5cm)$);
        \node[below=.1cm]at($(rtB)!0.5!(rtE)+(0,-1.5cm)$){RIPでの経路 【ホップ数\(=2\),総コスト\(=800\)】};
        \foreach \u in{A,B,C,D,E}{
                \node[below=.2cm,text=white]at(rt\u){ルータ\u};
            }
        \foreach \u in{A,B}{
                \node at(host\u){\small ホスト \u};
            }
    \end{tikzpicture}
    \caption{OSPFでの経路選択\cite[p.284]{マスタリングTCPIP}}
    \label{fig:OSPFでの経路選択}
\end{figure}
\subsection{VLAN}
VLAN(Virtual LAN)とは,1つのL2スイッチに接続されていても,異なるセグメントとして設定でき,VLANどうしの通信を一切遮断する技術である.
この技術を用いることで,ポートごとのにセグメントを分割でき,ブロードキャストドメインを区切ることができる.これにより,ネットワーク負荷を軽減でき,安全性が向上する(\figref{fig:VLAN}).
さらに,スイッチをまたいだVLANの構築(\figref{fig:スイッチをまたいだVLAN})の構築により,ネットワークの配線を変えずにネットワークの構造を変更でき,機器の節約にもつながる.
この技術はタグVLANと呼ばれ,VLAN IDを各セグメントに定義し,EthernetヘッダにVLAN IDを付加することで,どのセグメントにフレームを転送するか決める\cite[p.105\ -\ p.106]{マスタリングTCPIP}.
\begin{figure}
    \centering
    \begin{minipage}[b]{.55\textwidth}
        \centering
        \scalebox{.7}{
            \begin{tikzpicture}
                \tikzset{port/.style={rounded corners,draw,text width=.5cm,minimum height=.5cm,fill=#1!30,draw=#1!70!black}};
                \node[port=magenta](p1){};
                \node[port=magenta,right=.3cm of p1](p2){};
                \node[port=magenta,right=.3cm of p2](p3){};
                \node[port=cyan,right=.3cm of p3](p4){};
                \node[port=cyan,right=.3cm of p4](p5){};
                \node[port=cyan,right=.3cm of p5](p6){};
                \node[inner sep=2mm,fit={(p1)(p2)(p3)(p4)(p5)(p6)},draw,very thick](switch){};
                \node[above=.1cm]at(switch.north){スイッチ};
                \node[below=1cm of p3](sg13){\includegraphics[keepaspectratio,width=1cm]{../network_iconset/laptop_nb.pdf}};
                \node[left=.2cm of sg13](sg12){\includegraphics[keepaspectratio,width=1cm]{../network_iconset/laptop_nb.pdf}};
                \node[left=.2cm of sg12](sg11){\includegraphics[keepaspectratio,width=1cm]{../network_iconset/laptop_nb.pdf}};
                \node[below=1cm of p4](sg24){\includegraphics[keepaspectratio,width=1cm]{../network_iconset/laptop_nb.pdf}};
                \node[right=.2cm of sg24](sg25){\includegraphics[keepaspectratio,width=1cm]{../network_iconset/laptop_nb.pdf}};
                \node[right=.2cm of sg25](sg26){\includegraphics[keepaspectratio,width=1cm]{../network_iconset/laptop_nb.pdf}};
                \begin{scope}[on background layer]
                    \node[inner xsep=-1.1mm,inner ysep=3mm,fit={(sg11)(sg12)(sg13)(p1)(p2)(p3)},draw,rounded corners,fill=magenta!10,draw=magenta!70!black](sg1){};
                    \node[inner xsep=-1.1mm,inner ysep=3mm,fit={(sg24)(sg25)(sg26)(p4)(p5)(p6)},draw,rounded corners,fill=cyan!10,draw=cyan!70!black](sg2){};
                \end{scope}
                \foreach \u in {1,2,3}{
                        \draw[line width=1mm,draw=magenta!10,line cap=round](p\u.center)--(sg1\u.north);
                        \draw[ultra thick,line cap=round](p\u.center)--(sg1\u.north);
                    }
                \foreach \u in {4,5,6}{
                        \draw[line width=1mm,draw=cyan!10,line cap=round](p\u.center)--(sg2\u.north);
                        \draw[ultra thick,line cap=round](p\u.center)--(sg2\u.north);
                    }
                \node[below]at(sg1.south){\scriptsize セグメントA};
                \node[below]at(sg2.south){\scriptsize セグメントB};
                \node[above=.5cm of p5](pcap){\scriptsize ポート};
                \draw[-latex,draw=gray](pcap)--(p6);
                \draw[-latex,draw=gray](pcap)--(p5);
                \draw[-latex,draw=gray](pcap)--(p4);
                \node[above=.5cm of p2](pcap){\scriptsize ポート};
                \draw[-latex,draw=gray](pcap)--(p1);
                \draw[-latex,draw=gray](pcap)--(p2);
                \draw[-latex,draw=gray](pcap)--(p3);
            \end{tikzpicture}
        }
        \caption{VLAN}
        \label{fig:VLAN}
    \end{minipage}
    \begin{minipage}[b]{.43\textwidth}
        \centering
        \scalebox{.7}{
            \begin{tikzpicture}
                \newcommand{\laptop}{\includegraphics[keepaspectratio,width=.5cm]{../network_iconset/laptop_nb.pdf}}
                \tikzset{port/.style={rounded corners,draw,text width=.2cm,minimum height=.2cm,fill=#1!30,draw=#1!70!black}};
                \node[port=cyan](p11){};
                \node[port=cyan,right=.2cm of p11](p12){};
                \node[port=magenta,right=.8cm of p12](p13){};
                \node[port=magenta,right=.2cm of p13](p14){};
                \node[port=magenta,right=.2cm of p14](p15){};
                \node[draw,thick,inner sep=.5mm,fit={(p11)(p12)(p13)(p14)(p15)}](switch1){};
                \node[port=cyan,below=1cm of p11](p21){};
                \node[port=cyan,right=.2cm of p21](p22){};
                \node[port=magenta,right=.8cm of p22](p23){};
                \node[port=magenta,right=.2cm of p23](p24){};
                \node[port=magenta,right=.2cm of p24](p25){};
                \node[draw,thick,inner sep=.5mm,fit={(p21)(p22)(p23)(p24)(p25)}](switch2){};
                \foreach \u in{1,2,...,4}{
                        \node[below=.5cm of p2\u](lp2\u){\laptop};
                        \node[above=.5cm of p1\u](lp1\u){\laptop};
                    }
                \foreach \u in{1,2}{
                        \draw[line width=.6mm,draw=cyan!10,line cap=round](p1\u.center)--(lp1\u.south);
                        \draw[thick,line cap=round](p1\u.center)--(lp1\u.south);
                        \draw[line width=.6mm,draw=cyan!10,line cap=round](p2\u.center)--(lp2\u.north);
                        \draw[thick,line cap=round](p2\u.center)--(lp2\u.north);
                    }
                \foreach \u in{3,4}{
                        \draw[line width=.6mm,draw=magenta!10,line cap=round](p1\u.center)--(lp1\u.south);
                        \draw[thick,line cap=round](p1\u.center)--(lp1\u.south);
                        \draw[line width=.6mm,draw=magenta!10,line cap=round](p2\u.center)--(lp2\u.north);
                        \draw[thick,line cap=round](p2\u.center)--(lp2\u.north);
                    }
                \draw[line width=.6mm,draw=magenta!10,line cap=round](p15.center)--(p25.center);
                \draw[thick,line cap=round](p15.center)--(p25.center);
                \begin{scope}[on background layer]
                    \node[inner ysep=.1mm,fit={(lp11)(lp12)(lp21)(lp22)},draw,rounded corners,fill=cyan!10,draw=cyan!70!black,](sg1){};
                    \node[inner ysep=.1mm,fit={(lp13)(lp14)(lp23)(lp24)(p15)($(p25)+(.37cm,0)$)},draw,rounded corners,fill=magenta!10,draw=magenta!70!black](sg2){};
                \end{scope}
                \node[below]at(sg1.south){\tiny セグメントB};
                \node[below]at(sg2.south){\tiny セグメントA};
                \node[above]at(sg1.north){\tiny VLAN ID: 11};
                \node[above]at(sg2.north){\tiny VLAN ID: 10};
            \end{tikzpicture}
        }
        \caption{スイッチをまたいだVLAN}
        \label{fig:スイッチをまたいだVLAN}
    \end{minipage}
\end{figure}
\subsection{パケットフィルタ}
