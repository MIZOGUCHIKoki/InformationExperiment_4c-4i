\section{クラウドコンピューティング}\hypertarget{sec:クラウドコンピューティング}{}
\tocc
\subsection{クラウドとは}
\begin{frame}[t]{\ftitle}
    \begin{block}{クラウドコンピューティング}
        クラウドコンピューティング(以下,クラウド)とは,コンピュータリソース尾の利用形態.
        コンピュータの計算リソースやストレージ領域,アプリケーションによる処理をネットワーク経由で提供する.\hfill\cite{2015amazon}
    \end{block}
    クラウドは以下の4つに分類されることが多い.
    \begin{itemize}
        \setlength{\itemsep}{1em}
        \item IaaS(Infrastructure as a Service)
        \item PaaS(Platform as a Service)
        \item SaaS(Software as a Service)
        \item DaaS(Desktop as a Service)% VDIとの違い
    \end{itemize}
\end{frame}
\subsection{クラウドの分類}
\begin{frame}[t]{\ftitle}
    \begin{itemize}
        \item \textbf{IaaS}\\
              仮想サーバやストレージ,ネットワークサービスなどをインターネット経由で提供する.\\(AWS,Azure,GCP\footnote{Google Cloud Pratform},IBM Cloudなど)
        \item \textbf{PasS}\\
              ミドルウェアをサービスとして提供する.
              OSとミドルウェアの管理はサービス提供者で行われ,ユーザはミドルウェアを直接操作できる.\\(AWS,Azure,GCPなど)
        \item \textbf{SaaS}\\
              ソフトウェアをサービスとして提供する.\\(Slack,Zoom,Google Calendarなど)
    \end{itemize}
    \hfill\cite{2015amazon}
\end{frame}
\begin{frame}[t]{\ftitle}
    \begin{itemize}
        \item \textbf{DaaS}\\
              ``クラウドサービスとして提供される仮想デスクトップ環境を組織として利用するしかけ.''\hfill\cite{ISディジタル辞典DaaS}\\
              %   \begin{itemize}
              %       \item[{\bfseries\Large\cmark}] \underline{クライアントを安価に抑えることができる.}(クライアントが依頼した処理をサーバが処理するため)
              %       \item[{\bfseries\Large\xmark}] \underline{ネットワークネットワーク負荷が大きくなる.}(ディスクトップ環境を提供するため)
              %   \end{itemize}
              (Azure Virtual Desktopなど)\\
              \hrulefill
        \item \textbf{VDI}(Virtual Desktop Infrastructure)\\
              ``クライアントPCのディスクトップサーバ上に仮想化して集約し,稼働させる仕組み.''\hfill\cite{ISディジタル辞典VDI}
              \item[\textbf{\large !}]DaaSはVDIの一種である.
    \end{itemize}
\end{frame}
\subsection{代表的なIaaS}
\begin{frame}[t]{\ftitle}
    \begin{itemize}
        \setlength{\itemsep}{1em}
        \item \underline{AWS}
        \item Microsoft Azure
        \item Google Cloud Pratform
        \item さくらのクラウド(さくらインターネット)
    \end{itemize}
\end{frame}