\subsection{WebページのSSL化}
Let's Encrypt を用いてHTTPS通信を実現する.
\begin{enumerate}
    \item 必要なパッケージをインストールし,Python3をセットアップする.
          \begin{lstlisting}
$ sudo dnf install -y python3 augeas-libs pip
$ sudo python3 -m venv /opt/certbot/
$ sudo /opt/certbot/bin/pip install --upgrade pip
$ sudo /opt/certbot/bin/pip install certbot
$ sudo ln -s /opt/certbot/bin/certbot /usr/bin/certbot
\end{lstlisting}
    \item Webサーバを停止する.
          \begin{lstlisting}
$ sudo systemctl stop httpd
\end{lstlisting}
    \item \texttt{certbot}を実行し,質問に回答する.これにより,証明書が作成される.
          \begin{lstlisting}
$ sudo certbot certonly --standalone
\end{lstlisting}
          \begin{itemize}
              \item 証明書の期限切れを通知するメールアドレス:\texttt{250373b@ugs.kochi-tech.ac.jp}
              \item 利用規約に同意するか:\texttt{Y}
              \item メールアドレスを共有するか?:\texttt{N}
              \item 証明書に記載するドメイン名:\texttt{c5.exp.info.kochi-tech.ac.jp}
          \end{itemize}
          完了したら,証明書が\texttt{/etc/letsencrypt/live/c5.kochi-tech.ac.jp}下に作成される.
    \item \texttt{/etc/httpd/conf.d/ssl.conf}へ以下を追加し,証明書などのファイル所在を記す.
          \begin{lstlisting}[style=file,caption={\ttfamily /etc/httpd/conf.d/ssl.conf}]
%\ryaku%
<VirtualHost _default_:443>
 ErrorLog logs/ssl_error_log
 TransferLog logs/ssl_access_log
 LogLevel warn
 SSLEngine on
 ServerName      c5.exp.info.kochi-tech.ac.jp
 DocumentRoot    /var/www/html
 SSLHonorCipherOrder on
 SSLCipherSuite PROFILE=SYSTEM
 SSLProxyCipherSuite PROFILE=SYSTEM
 SSLCertificateFile      /etc/letsencrypt/live/c5.exp.info.kochi-tech.ac.jp/cert.pem
 SSLCertificateKeyFile   /etc/letsencrypt/live/c5.exp.info.kochi-tech.ac.jp/privkey.pem
 SSLCertificateChainFile /etc/letsencrypt/live/c5.exp.info.kochi-tech.ac.jp/chain.pem        
 %\ryaku%
</VirtualHost>
%\ryaku%
    \end{lstlisting}
    \item \texttt{httpd}を起動する.
          \begin{lstlisting}
$ sudo systemctl start httpd
\end{lstlisting}
    \item \url{https://c5.exp.info.kochi-tech.ac.jp}にアクセスし,SSLの適用を確かめる.
\end{enumerate}