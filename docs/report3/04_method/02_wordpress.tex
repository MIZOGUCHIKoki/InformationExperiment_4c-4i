\subsection{WordPressのインストールとWebページの公開}
WordPressを用いたWebページを公開する.
\begin{enumerate}
    \item WordPressに必要なパッケージをインストールする.
          \begin{lstlisting}
$ sudo dnf install wget php-mysqlnd httpd php-fpm php-mysqli mariadb105-server php-json php php-devel -y
\end{lstlisting}
    \item データベース,Webサーバを起動する.
          \begin{lstlisting}
$ sudo systemctl start mariadb httpd
\end{lstlisting}
    \item \texttt{wget}コマンドを利用して,WordPressのパッケージをダウンロード,解凍し,所定の場所にコピーする.
          そして,\texttt{/var/www/wordpress}のディレクトリ管理権限を\texttt{apache}に変更する.
          \begin{lstlisting}
$ sudo cd /tmp
$ sudo wget http://ja.wordpress.org/latest.tar.gz
$ sudo tar -xvf latest.tar.gz
$ sudo cp -r wordpress /var/www/
$ sudo cp -r wordpress/* /var/www/html/
$ sudo chown -R apache.apache /var/www
    \end{lstlisting}
    \item \texttt{httpd}の設定ファイルを更新する.
          \begin{itemize}
              \item \texttt{DocumentRoot}を\texttt{/var/www/html}に変更.
              \item \texttt{/var/www/html}のディレクトリに,\texttt{.htasccess}によるディレクティブ上書きを許可する.
                    さらに,ファイル名を指定しないアクセスに対しては\texttt{index.php}を返す設定をする.
          \end{itemize}
          \begin{lstlisting}[style=file,caption={\ttfamily /etc/httpd/conf/httpd.conf}]
%\ryaku%
DocumentRoot %{\color{red}"/var/www/html"}%
%\ryaku%
<Directory "/var/www/html">
            AllowOverride %{\color{red}All}%
%\ryaku%
<IfModule dir_module>
            DirectoryIndex index.php
</IfModule>
%\ryaku%
          \end{lstlisting}
    \item データベースにアクセスして,WordPress用のユーザを作成する.作成したユーザに対して,完全な権限を付与する.
          \begin{lstlisting}
$ sudo mysql -u root -p
> CREATE USER 'wordpress-user'@'localhost' IDENTIFIED BY '******';
> CREATE DATABASE `wordpress-db`;
> GRANT ALL PRIVILEGES ON `wordpress-db`.* TO "wordpress-user"@"localhost";
> FLUSH PRIVILEGES;
\end{lstlisting}
    \item \url{https://api.wordpress.org/secret-key/1.1/salt/}にアクセスして,ランダムに生成されるキーセット値を取得する.
    \item \texttt{/var/www/html/wp-config.php}ファイルでデータベース情報と,キーセットを登録する.
          \begin{lstlisting}[style=file,caption={\ttfamily /var/www/html/wp-config.php},language=php,escapechar={}]
define( 'DB_NAME', 'wordpress-db' );
define( 'DB_USER', 'wordpress-user' ); 
define( 'DB_PASSWORD', '******' );
define( 'DB_HOST', 'localhost' ); 
define( 'DB_CHARSET', 'utf8' );
define( 'DB_COLLATE', '' );
define('AUTH_KEY',         '**secret**');
define('SECURE_AUTH_KEY',  '**secret**');
define('LOGGED_IN_KEY',    '**secret**');
define('NONCE_KEY',        '**secret**');
define('AUTH_SALT',        '**secret**');
define('SECURE_AUTH_SALT', '**secret**');
define('LOGGED_IN_SALT',   '**secret**');
define('NONCE_SALT',       '**secret**');
\end{lstlisting}
    \item \texttt{/var/www}以下のディレクトリに対して,パーミッションを変更し,SGIDを付与する.
          \begin{lstlisting}
$ sudo chmod 2775 /var/www
$ find /var/www -type d -exec sudo chmod 2775 {} \;
$ find /var/www -type f -exec sudo chmod 0644 {} \;
    \end{lstlisting}
    \item \texttt{httpd}を再起動する.
          \begin{lstlisting}
$ sudo systemctl restart httpd
\end{lstlisting}
    \item \url{c5.exp.info.kochi-tech.ac.jp}にアクセスして,WordPressが表示されるか確認する.
    \item サイトのタイトル,ユーザ名を設定して,WordPress内で記事を編集し,公開する.
\end{enumerate}