\section{目的}
本実験では,VLANとルータを用いて,ネットワークを分割する.また,所属するネットワークに対してNAPTを設定し,特定の通信以外を拒否し,加えてポートフォワーディングを設定する.
\subsection{構築する理由}
昨今,インターネットが普及し,家庭でもLANを持つことが普通になっている.たとえば,集合住宅でインターネット回線を引くことを考える.建物全体で1つのLANとし,各戸間でのネットワークを分離しない場合,別戸のネットワーク機器へのアクセスは容易である.各戸間でネットワークの分離することにより,各戸間でのやりとりに制限を設けられる.\par
また,1つのスイッチ内に仮想的なスイッチを複数用意し,各ネットワークごとにL2スイッチを設置せずとも,簡便にLANを構築できる.\par
さらに,IPv4アドレスの枯渇問題に対応すべく,プライベートIPアドレスとグローバルIPアドレスを対応づけて通信し,外部から直接LAN内ホストへのアクセスを禁止する.加えて,許可した通信以外を拒否する設定をし,LANの安全性を高める.
\subsection{構築するものの概要}
LAN間のネットワーク分離はルータを用いる.
1台のスイッチ内に仮想的なスイッチを用意する技術をVLAN(Virtual LAN)と呼び,本実験ではこれを採用する.\par
さらにプライベートIPアドレスとグローバルIPアドレスをポート番号をもとに動的に対応づけるしくみとしてNAPT,
特定通信以外を拒否するしくみとしてパッケットフィルタリング,
特定ポートの通信を特定ホストに転送するポートフォワーディングを採用する.
ポートフォワーディングを設定することにより,パケットフィルタリングを実現できる.