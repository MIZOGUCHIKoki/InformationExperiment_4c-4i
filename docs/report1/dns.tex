\chapter{DNS}
\section{DNS誕生の背景と実験の目的}
この実験では,DNSを用いてLAN内のコンピュータを,ホスト名で識別できるようにする.DNSを用いることで,IPアドレスではなくアルファベットや記号を用いたホスト名でのアクセスが可能になる.\par
DNSが誕生する以前,ARPAnet\footnote{米国の重要な機関を結ぶ広域コンピュータネットワーク\cite[p.1]{DNSBIND}.}は数百台のホストだけで構成される小さなネットワークであり,
ARPAnet上すべてのホスト情報は,\texttt{HOSTS.TXT}というファイルで管理されていた.
\texttt{HOSTS.TXT}の更新は,SRI-NIC\footnote{SIRのネットワークインフォメーションセンター(NIC)のホスト.SIRは元のStanford Research Institute.\texttt{HOSTS.TXT}はSIRのNICによって管理されていた.}にFTP接続して行われており,
ARPAnetの管理者は,ホスト情報に変更があるとSRI-NICへ電子メールを用いて通知していた.
しかし,ARPAnetの規模が大きくなると,当然\texttt{HOSTS.TXT}のサイズが大きくなり,SRI-NICから更新情報を得ようとするホスト数も増加する\cite[p.3]{DNSBIND}.
ARPAnetがTCP/IPプロトコルに移行すると,ネットワーク上のホスト数が爆発的に増え,\texttt{HOSTS.TXT}での管理が困難になった.その主な原因は3つある.
\begin{enumerate}
      \item \textbf{トラフィックと負荷的問題}\\
            \texttt{HOSTS.TXT}配布に伴うプロセッサの負荷と,ネットワークトラフィックの増大にSRI-NICが耐えきれなかった.
      \item \textbf{名前の衝突問題}\\
            \texttt{HOSTS.TXT}は,同一ホスト名を登録できないが,ホスト名割り当てに権限がないNICはそれを防げなかった.
      \item \textbf{一貫性担保の問題}\\
            \texttt{HOSTS.TXT}の一貫性担保は困難であった.ARPAnetの管理者が新しい\texttt{HOSTS.TXT}に更新するころには,ホストのアドレスが変わっていたり,新しいホストが追加されたりした.
\end{enumerate}
\hfill\cite[p.4]{DNSBIND}\par
ここで登場するのが,DNSである.DNSは分散データベースであり,\texttt{HOSTS.TXT}での管理とは異なり,データベースの各部分はローカルに管理されながら,それぞれの部分データはネットワーク全体で利用できる\cite[p.5]{DNSBIND}.
今回の実験では,DNSを提供するDNSサーバを構築し,ホスト名を用いた通信を実現する.

\begin{wrapfigure}{r}[0mm]{.5\textwidth}
      \centering
      \caption{高知工科大学情報学群のWEBページURIとラベル名}
      \label{fig:KUTINFOWebページ}
      \input{01_url}
\end{wrapfigure}
\section{DNSとは}
ホスト名を用いた通信を実現するために,サーバに名前解決システムをDNSを用いて構築する.
ここで高知工科大学情報学群ホームページのURIを例にあげる.URI冒頭の\texttt{www}はホスト名である.
DNSはUNIXファイルシステムと同様に木構造を持っている.ファイルシステムの根(root)は\texttt{\ /\ }であるのに対して,DNSの根(ルートDNS)は空ラベル(``'')で表現される.
UNIXファイルシステムにおける木構造の各節は,「ディレクトリ」であり,DNSにおいてはドメインである.
DNSでは木構造全体の,ルートまでのラベルをドット(``\texttt{.}'')で連結したものがDNSの絶対名となる.これをFQDN(Full Qualified Domain Name)と呼ぶ.
当然,FQDNは世界に同じものが存在しないように作成する必要がある.\par
DNSでは,各ドメインをいくつかのサブドメインに分割し,サブドメインの管理をサブドメイン内で完結できる.
たとえば,ドメイン\texttt{info.kochi-tech.ac.jp}において,\texttt{jp}のサブドメインである\texttt{ac}は,\texttt{ac.jp}で終わるドメインについて責任を持つ.
また,\texttt{ac}は,\texttt{kochi-tech}というサブドメインをもち,\texttt{kochi-tech.ac.jp}が末尾につくドメイン名の責任を高知工科大学に委任している.
ここで,ゾーンという概念が登場する.ゾーンとは,名前空間において独立し,自治管理される部分である.