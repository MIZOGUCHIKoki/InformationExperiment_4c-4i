\section{目的}
本実験では,今までオンプレミスで行った作業を,クラウド上で実現する.
クラウド化する目的は大きく2つある.\par
第一に,クラウド化することで,管理コストや運用コストを抑えられる.
クラウドのサービスにもよるが,オンプレミスでは,サーバ本体を用意し,サーバ本体の管理もしなければならない.
サーバ本体にかかるコストもあるが,サーバの管理には電気代や故障した場合の部品代など,運用にもコストが大きくかかる.
クラウド化によって,これらのコストは不要になり,管理の手間もクラウドサービスに任せられる.\par
第二に,災害対策やセキュリティ対策に対しても,クラウド化による大きな利点がある.
オンプレミスでは,当然セキュリティ対策や災害対策も含めて行わねばならない.
これにも前述したコストや専門的な知識が必要であり,対策が不十分な場合は損害が発生することもある.
クラウドサービスにもよるが,大概のクラウドサービスでは,セキュリティの専門知識がなくても高いセキュリティレベルを実現できる.
さらに,国内外の遠隔地にデータを分散できるクラウドサービスは,災害対策にもつながる.\par
