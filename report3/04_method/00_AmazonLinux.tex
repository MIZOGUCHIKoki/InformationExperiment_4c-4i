\subsection{EC2インスタンスの作成と諸設定}
EC2インスタンスを,\tblref{tbl:インスタンスの作成とネットワークの設定}のもとで作成する.
Elastic IPアドレスの割り当てで,パブリックIPアドレスを固定化する.
また,\tblref{tbl:ファイアウォールの設定}に沿って,ファイアウォールを設定する.
外部への,任意のアクセスを許可し,外部からのアクセスは,指定した一部のアクセスのみを許可する.\par
\begin{table}[htbp]
    \centering
    \caption{インスタンスの作成とネットワークの設定}
    \label{tbl:インスタンスの作成とネットワークの設定}
    \begin{tabularx}{\textwidth}{lR}
        \toprule
        \multicolumn{1}{c}{設定内容} & \multicolumn{1}{c}{設定値}                  \\
        \midrule
        名前とタグ                    & \texttt{c5}                              \\
        OS                       & \texttt{Amazon Linux 2023 AMI}           \\
        アーキテクチャ                  & 64ビット (x86)                              \\
        インスタンスタイプ                & \texttt{t2.micro}                        \\
        キーペア                     & RSA\ (\texttt{.pem})                     \\
        ストレージ                    & \texttt{1\ x\ 8 GiB\ gp3}ルートボリューム(暗号化なし) \\
        Public IPの割り当て           & 有効化                                      \\
        セキュリティグループ               & セキュリティグループを作成する                          \\
        Elastic IPアドレスの割り当て      & 有効化(\texttt{3.94.71.105})                \\
        ネットワークボーダーグループ           & \texttt{us-east-1}                       \\
        パブリックIPアドレスルプール          & AmazonのIPv4アドレスプール                       \\
        \bottomrule
    \end{tabularx}
\end{table}
\begin{table}[htbp]
    \centering
    \caption{ファイアウォールの設定}
    \label{tbl:ファイアウォールの設定}
    \begin{tabularx}{\textwidth}{cRRR}
        \toprule
        ルール                        & \multicolumn{1}{c}{プロトコル}     & \multicolumn{1}{c}{ポート範囲} & \multicolumn{1}{c}{送信先} \\
        \midrule
        アウトバウンドルール                 & すべて                           & すべて                       & \texttt{0.0.0.0/0}      \\
        \bottomrule
        \toprule
        ルール                        & \multicolumn{1}{c}{タイプ/プロトコル} & \multicolumn{1}{c}{ポート範囲} & \multicolumn{1}{c}{ソース} \\
        \midrule
        \multirow{6}{*}{インバウンドルール} & SSH/TCP                       & 21                        & \texttt{0.0.0.0/0}      \\
                                   & カスタムTCP/TCP                   & 22                        & \texttt{0.0.0.0/0}      \\
                                   & HTTP/TCP                      & 80                        & \texttt{0.0.0.0/0}      \\
                                   & HTTPS/TCP                     & 443                       & \texttt{0.0.0.0/0}      \\
                                   & カスタムTCP/TCP                   & 8080                      & \texttt{0.0.0.0/0}      \\
                                   & カスタムTCP/TCP                   & 60000\ -\ 60010           & \texttt{0.0.0.0/0}      \\
        \bottomrule
    \end{tabularx}
\end{table}
インスタンスを作成すると,鍵ファイル\texttt{c5.pem}のダウンロードが始まる.
このファイルは厳重に管理する.
また,Elastic IPアドレスを,DNSのAレコードに記載し,Elastic IPとドメイン名\texttt{c5.exp.info.kochi-tech.ac.jp}を対応づける.
