\subsection{SSLを用いたリモートログインの実現}
SSH接続するClientは,高知工科大学ワークステーションのコンピュータを利用する.
\begin{enumerate}
  \item ダウンロードされた鍵ファイル\texttt{c5.pem}ファイルを\texttt{\textasciitilde/.ssh}に移動させる.
  \item \texttt{\textasciitilde/.ssh}がない場合は,\texttt{mkdir}コマンドを用いてディレクトリを作成する.
  \item \texttt{\textasciitilde/.ssh/config}ファイルがない場合は,\texttt{touch}コマンドを用いてファイルを作成して,以下の内容を書き込み,保存する.
        この設定により,\texttt{c5.exp.info.kochi-tech.ac.jp}のホストへアクセスするときに,ユーザ名,鍵ファイルの場所,プロキシの設定を省略できる.
        インスタンス作成時に,デフォルトで作成されるユーザ名は\texttt{ec2-user}である.
        高知工科大学のプロキシは,\texttt{http://proxy.noc.kochi-tech.ac.jp:3128}である.
        \begin{lstlisting}[style=file,caption={\ttfamily \textasciitilde/.ssh/config},escapechar={}]
Host c5.exp.info.kochi-tech.ac.jp
 User ec2-user
 IdentityFile ~/.ssh/c5.pem
 ProxyCommand nc -X connect -x proxy.noc.kochi-tech.ac.jp:3128 %h %p
        \end{lstlisting}
  \item 以下のコマンドを実行し,インスタンスへSSH接続できるか確認する.
        \begin{lstlisting}
$ sudo chmod 600 ~/ssh/c5.pem
$ ssh c5.exp.info.kochi-tech.ac.jp
          \end{lstlisting}
        以下の出力を確認できたら,SSHによるリモートログインに成功している.
        \begin{lstlisting}
A newer release of "Amazon Linux" is available.
  Version 2023.1.20230725:
Run "/usr/bin/dnf check-release-update" for full release and version update info
   ,     #_
   ~\_  ####_        Amazon Linux 2023
  ~~  \_#####\
  ~~     \###|
  ~~       \#/ ___   https://aws.amazon.com/linux/amazon-linux-2023
   ~~       V~' '->
    ~~~         /
      ~~._.   _/
         _/ _/
       _/m/'
Last login: Thu Aug  3 14:25:50 2023 from xxx.xxx.xxx.xxx
[ec2-user@ip-172-31-93-16 ~]$ 
\end{lstlisting}
\end{enumerate}
