\subsection{Dockerのインストールとコンテナの構築}
\begin{enumerate}
    \item Dockerのパッケージをインストールし,有効化する.権限も更新する.
          \begin{lstlisting}
$ sudo dnf install -y docker
$ sudo systemctl enable --now docker
$ sudo systemctl status docker
$ sudo usermod -aG docker ec2-user
\end{lstlisting}
    \item Dockerのコンテナを作成する.ここではイメージをレジストリから取得する.
          今回は,\texttt{httpd}のイメージを利用して,8080番ポートでHTTP通信を実現する.
          \begin{lstlisting}
$ docker pull httpd
$ docker images
REPOSITORY   TAG       IMAGE ID       CREATED       SIZE
httpd        latest    d140b777a247   3 weeks ago   168MB
          \end{lstlisting}
    \item コンテナを作成する.8080番ポートに来たアクセスをコンテナ内の80番ポートにフォワーディングする.
          \begin{lstlisting}
$ docker run httpd -itd -p 8080:80
\end{lstlisting}
    \item  作成したコンテナが動いているか確認する.
          \begin{lstlisting}
$ docker ps 
CONTAINER ID   IMAGE     COMMAND              CREATED         STATUS         PORTS                                   NAMES
0fd95973b539   httpd     "httpd-foreground"   8 minutes ago   Up 8 minutes   0.0.0.0:8080->80/tcp, :::8080->80/tcp   wizardly_galois
\end{lstlisting}
    \item Dockerのコンテナへ,Webアクセスできることを確認する.
          \url{http://c5.exp.info.kochi-tech.ac.jp:8080}へアクセスし,「\texttt{It Works!}」の文字を確認する.
\end{enumerate}