\section{内容}
クラウドサービスを用いて,以下のことを実現する.
\begin{enumerate}
    \renewcommand{\labelenumi}{\theenumi.\ }
    \renewcommand{\labelenumi}{\theenumi.\ }
    \begin{multicols}{2}
        \item Public IPアドレスの取得と,DNSへの登録.
        \item ファイアウォールの設定.
        \item クラウドサービスへのリモートログイン.
        \item WordPressを用いたWebページ公開.
        \item WebページにBasic認証を適用する.
        \item WebページのSSL化によるHTTPS通信の実現.
        \columnbreak
        \item Dockerをクラウドにインストールし,\texttt{httpd}のイメージに対するコンテナを立ち上げる.
        コンテナを立ち上げると,8080ポートでコンテナ内のWebサーバへアクセスできるようにする.
    \end{multicols}
    \item \TeX ファイルをコンパイルするシステムを構築する(\figref{fig:処理の流れ}).
          \begin{enumerate}
              \renewcommand{\labelenumii}{\theenumii. }
              \item FTPクライアントから,クラウドへFTP接続し,所定の場所へ\TeX ファイルを置く.
              \item クラウドが\TeX ファイルの更新を検知し,Dockerのコンテナへ\TeX ファイルを送る.
              \item クラウド上のDockerコンテナ内で,受信した\TeX ファイルの更新を検知し,コンパイルしてPDFを生成する.
              \item コンテナ内で生成したPDFファイルを,コンテナ外から取得する.
              \item 生成されたPDFをFTPクライアントから取得する.
          \end{enumerate}
\end{enumerate}
\begin{figure}[htbp]
    \centering
    \begin{tikzpicture}
        \tikzset{file/.style={rectangle,minimum width=1.3cm,minimum height=1.5cm,draw,fill=gray!10}};
        \tikzset{machine/.style={inner sep=3mm,rounded corners,draw,very thick,fit={#1}}}
        \node[file]at(0,0)(main1){\scriptsize\ttfamily main.tex};
        \node[file,right=3cm of main1](main2){\scriptsize\ttfamily main.tex};
        \node[file,right=3cm of main2](main3){\scriptsize\ttfamily main.tex};
        \node[file,below=1cm of main3](pdf1){\scriptsize\ttfamily main.pdf};
        \node[file,left=3cm of pdf1](pdf2){\scriptsize\ttfamily main.pdf};
        \node[file,left=3cm of pdf2](pdf3){\scriptsize\ttfamily main.pdf};
        \node[machine=(main3)(pdf1),dotted](docker){};
        \node[above]at(docker.north)(dockercap){\small コンテナ};
        \node[machine=(main2)(main3)(docker)(dockercap)(pdf1)](ec2){};
        \node[below]at(ec2.south)(cloudcap){\small クラウド};
        \draw[thick,-latex](main1.east)to[bend left=30]node[midway,above]{\small\ttfamily FPT put}(main2.west);
        \draw[thick,-latex](main2.east)to[bend left=30]node[midway,above]{\small\ttfamily docker cp}(main3.west);
        \draw[thick,-latex](pdf1.west)to[bend left=30]node[midway,below]{\small\ttfamily docker cp}(pdf2.east);
        \draw[thick,-latex](pdf2.west)to[bend left=30]node[midway,below]{\small\ttfamily FTP get}(pdf3.east);
        \node at(main1.south |- cloudcap.west){\small FTPクライアント};
        \node[machine=(main1)(pdf3),dashed,thick]{};
        \draw[-Stealth,thick](main3)--(pdf1)node[midway,fill=white,draw]{\scriptsize\ttfamily uplatex{\normalfont など}};
    \end{tikzpicture}
    \caption{処理の流れ}
    \label{fig:処理の流れ}
\end{figure}