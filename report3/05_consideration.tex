\section{考察}
オートスケーリングとその問題について取り上げる.
オートスケーリング(オートスケール)とは,\ref{sec:要素技術}節で解説したスケーリングを,自動で行うしくみである.
AWSでのオートスケーリングは以下のように説明されている.
\begin{quote}
    ``AWS Auto Scaling は、安定した予測可能なパフォーマンスを可能な限り低コストで維持するためにアプリケーションをモニタリングし、容量を自動で調整します。''\hfill\cite{autoscaling}
\end{quote}
オートスケーリングでは,利用状況に応じて,スケールアップ,スケールダウン,スケールアウト,スケールインする.
オートスケーリングを用いることで,仮想化の「コストや管理の手間を削減できる」というメリット最大限引き出せる.
オートスケーリングのデメリットを説明するために,「ステートレス」と「ステートフル」の言葉を導入する.
\begin{description}
    \item[ステートレス] ``利用者の状態をシステムが保持しており、その内容に応じて処理結果を変えること。''
    \item[ステートフル] ``利用者の状態をシステムの内部に保持せず、入力された値だけを使用して処理すること。同じ入力に対しては、常に同じ結果が得られる。''
\end{description}
\begin{flushright}
    \cite[p.64]{IT用語図鑑}
\end{flushright}
ここで,オートスケーリングでスケールインやスケールアウトした場合を考える.
ステートレスでは,サーバの増減にかかわらず,どのサーバでも処理を引き継げる.これは,ユーザの状態をシステムが保持していないことに起因する.
しかし,ステートフルの場合は,ユーザの入力状況によってサーバの処理が変わる.スケールインやスケールアウトしたとき,ユーザの状態がほかのサーバに引き継がれていない場合は障害が生じる.
これを解決するためには,ユーザの状態をサーバ間で共有するしくみが必要だ.
加えて,オートスケーリングによるスケールインでは,サーバが「削除」されるので,サーバにデータが保存される場合はその注意が必要だ.
予期せぬデータの消失を防ぐために,サーバを削除する前にスケールアウトで作成したサーバと,削除前の差分をバックアップとして自動的に保存する方法を提案する.
これにより,万が一データが消失した場合でもデータの復元が簡単であるほか,差分で保存することによりバックアップファイルサイズを小さくできる.
