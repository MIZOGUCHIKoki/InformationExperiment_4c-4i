\section{Docker(Option)}\label{chap:docker}
\tocc
\subsection{コンテナ立ち上げと利用}
\begin{frame}[t]{\ftitle}
    \begin{itemize}
        \setlength{\itemsep}{1em}
        \item コンテナを利用する.
              \begin{enumerate}
                  \setlength{\itemsep}{.5em}
                  \item イメージを作成(\texttt{build})または,取得(\texttt{docker pull})する.\\
                        イメージの作成には\texttt{Dockerfile} を書く.
                  \item コンテナを作成して起動(\texttt{run})する.
                  \item コンテナ内でコマンドを実行(\texttt{exec})する.
              \end{enumerate}
        \item コンテナ利用後.
              \begin{enumerate}
                  \setlength{\itemsep}{.5em}
                  \item コンテナを停止(\texttt{stop})する.
                  \item コンテナを削除(\texttt{rm})する.
                  \item イメージを削除(\texttt{rmi})する.
              \end{enumerate}
    \end{itemize}
\end{frame}
\subsection{イメージ作成・取得}
\begin{frame}[containsverbatim,t]{\ftitle}
    \begin{itemize}
        \item \texttt{Dockerfile}
              \begin{lstlisting}[language=docker-compose-2]
FROM mysql:latest
ENV MYSQL_ROOT_PASSWORD root00
            \end{lstlisting}
        \item レジストリからイメージを取得(Pull)する.
              \begin{lstlisting}[backgroundcolor={\color{black}},basicstyle={\color{white}\ttfamily},]
$ docker pull mysql:latest
            \end{lstlisting}
    \end{itemize}
\end{frame}
\subsection{実演}
\begin{frame}{\ftitle}
    \includegraphics[keepaspectratio,width=\textwidth]{docker_logo.png}\\\vspace{1em}
    \hfill\cite{docker}
\end{frame}