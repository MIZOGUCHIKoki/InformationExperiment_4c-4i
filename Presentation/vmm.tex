\section{仮想化技術と仮想マシン}
\tocc
\subsection{仮想化と仮想マシン}
\newcommand{\ie}{\tikz{\draw[arrows={-Latex[line width=1pt,fill=white,length=10pt]}](0,0)--(1,0)}\ }
\begin{frame}[t]{\ftitle}
    \newcommand{\server}{サーバー}
    \begin{block}{仮想化}
        ``コンピューターの物理的資源を論理的に分割して,それぞれ独立並列した状態で利用できるようにすること.
        1台の\server で,複数の基本ソフトを独立並列に動作させる\server 仮想化など.''\hfill\cite{スーパー大辞林}
    \end{block}
    \ie 1台の計算機で複数のOSやアプリケーションなどを並列に動作させること.
    \vspace{.5em}
    \begin{block}{Virtual Machine(仮想機械,仮想マシン)}
        ``あるコンピューターシステムの動作を,別システムで再現するソフトウエア.また,そのような動作環境.あるOSの動作を別のOS上で再現する場合など.バーチャルマシン.VM.''\hfill\cite{スーパー大辞林}
    \end{block}
    \ie 仮想化するためのソフトウェアや動作環境.
\end{frame}
\begin{frame}{\ftitle}
    \begin{figure}
        \centering
        \begin{minipage}[b]{.48\textwidth}
            \centering
            \begin{tikzpicture}
                \node[str,fill=gray!60,text width=.42\textwidth](hw1){物理サーバ};
                \node[str,fill=gray!40,text width=.42\textwidth,above=.1cm of hw1](os1){OS};
                \node[str,fill=gray!20,text width=.42\textwidth,above=.1cm of os1](sw1){\scriptsize アプリケーション};
                \node[str,fill=gray!60,text width=.42\textwidth,right=.1cm of hw1](hw2){物理サーバ};
                \node[str,fill=gray!40,text width=.42\textwidth,above=.1cm of hw2](os2){OS};
                \node[str,fill=gray!20,text width=.42\textwidth,above=.1cm of os2](sw2){\scriptsize アプリケーション};
                \node[inner sep=.05cm,draw,rounded corners,fit={(hw1)(os1)(sw1)}](warp1){};
                \node[inner sep=.05cm,draw,rounded corners,fit={(hw2)(os2)(sw2)}](warp2){};
            \end{tikzpicture}
            仮想化前
        \end{minipage}
        \begin{minipage}[b]{.48\textwidth}
            \centering
            \begin{tikzpicture}
                \node[str,fill=gray!60,text width=.85\textwidth](hw1){ハードウェア};
                \node[str,fill=gray!50,text width=.85\textwidth,above=.1cm of hw1](vmm){仮想化ソフトウェア};
                \node[str,fill=gray!40,text width=.38\textwidth,above=.1cm of vmm.north east,anchor=south east](os1){OS};
                \node[str,fill=gray!40,text width=.38\textwidth,above=.1cm of vmm.north west,anchor=south west](os2){OS};
                \node[str,fill=gray!20,text width=.38\textwidth,above=.1cm of os1](sw1){\tiny アプリケーション};
                \node[str,fill=gray!20,text width=.38\textwidth,above=.1cm of os2](sw2){\tiny アプリケーション};
                \node[inner sep=.05cm,draw,rounded corners,fit={(hw1)(vmm)},draw=white](warp1){};
            \end{tikzpicture}
            仮想化後
        \end{minipage}
    \end{figure}
\end{frame}
\subsection{仮想化の歴史と展望}
\begin{frame}{\ftitle}
    \begin{figure}
        \centering
        \begin{tikzpicture}
            \tikzset{txt/.style={rounded corners,draw,text width=.9\textwidth,align=center}};
            \node[txt](v1){2000年代の仮想化導入目的は\ \textbf{\color{red}コスト削減}};
            \newcommand{\ga}{が}
            \node[txt,below=1cm of v1](v2){仮想化の利用\ga 一巡する};
            \node[txt,below=1cm of v2](v3){本来持つ利点\ga 認識される};
            \node[txt,below=1cm of v3](v4){仮想化の\textbf{\color{red}俊敏性}・\textbf{\color{red}迅速性}・\textbf{\color{red}柔軟性}をビジネスに活かす};
            \foreach \u \v in {v1/v2,v2/v3,v3/v4}{
                    \draw[line width=.2cm,-latex](\u)--(\v);
                }
            \node[right=.1cm]at($(v1)!0.5!(v2)$){\scriptsize サイロ化\footnotemark の課題を引き継いでしまう};
        \end{tikzpicture}
    \end{figure}
    \footnotetext{複数のサーバが乱立し,他のシステムとの連携が取れない状態.}
\end{frame}
\begin{frame}[t]{\ftitle}
    \begin{figure}
        \centering
        \begin{tikzpicture}
            \tikzset{rc/.style={text width=.3\textwidth,minimum height=2cm,rounded corners,align=center}};
            \tikzset{p/.style={align=center,rounded corners,fill=blue!60,minimum height=2cm,anchor=west,text=white,text width=.5cm,}};
            \tikzset{sfit/.style={draw=blue!60,inner sep=-.1mm,rounded corners}};
            \node[rc](p1){コスト削減};
            \node[p,left=0cm of p1.west](p){目\\的};
            \node[fit={(p)(p1)},sfit](pp1){};
            \node[above=.1cm of pp1](1t){\LARGE 一巡};
            \node[rc,below=.5cm of p1](a1){\small オンプレミスで構築された物理サーバを仮想サーバに集約};
            \node[p,left=0cm of a1](a){\rotatebox{90}{Action}};
            \node[fit={(a)(a1)},sfit](aa1){};
            \node[fit={(1t)(aa1)(pp1)},draw,rounded corners](warp1){};

            \node[rc,right=2.5cm of p1](p2){仮想化の俊敏性,迅速性,柔軟性をビジネスに活かす};
            \node[p,left=0cm of p2.west](p){目\\的};
            \node[fit={(p)(p2)},sfit](pp2){};
            \node[above=.1cm of pp2](2t){\LARGE 二巡};
            \node[rc,below=.5cm of p2,align=left](a2){\scriptsize ・システム開発の短縮化};
            \node[p,left=0cm of a2](a){\rotatebox{90}{Action}};
            \node[fit={(a)(a2)},sfit](aa2){};
            \node[fit={(2t)(aa2)(pp2)},draw,rounded corners](warp2){};
            \draw[-latex,line width=.2cm](warp1.east)--(warp2.west);
        \end{tikzpicture}
    \end{figure}
    \hfill\cite{仮想化歴史}
\end{frame}
\subsection{メリットとデメリット}
\begin{frame}[t]{\ftitle}
    \begin{minipage}[t]{.49\textwidth}
        \textbf{\cmark メリット}\\\hspace{1\zw}(仮想化導入の目的)
        \begin{itemize}
            \setlength{\itemsep}{1em}
            \item 無駄なリソースの削減
            \item コスト削減
                  \begin{itemize}
                      \item[\faArrowCircleORight] ハードウェアを減らせる
                  \end{itemize}
            \item リソースを集約できる
                  \begin{itemize}
                      \item[\faArrowCircleORight] 管理が簡便になる
                  \end{itemize}
        \end{itemize}
    \end{minipage}
    \begin{minipage}[t]{.49\textwidth}
        \textbf{\xmark デメリット}\\
        \begin{itemize}
            \setlength{\itemsep}{1em}
            \item 物理サーバに比べて性能が劣る.
            \item 物理層の故障による障害範囲が広くなる.
        \end{itemize}
    \end{minipage}
\end{frame}