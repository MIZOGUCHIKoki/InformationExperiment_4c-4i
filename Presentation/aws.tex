\section{AWS}
\tocc
\subsection{AWSとは}
\begin{frame}[t]{\ftitle}
    \begin{block}{AWS}
        Amazon Web Service の略称.Amazonが提供するクラウドサービスで,ネットワークを経由して仮想コンピュータやストレージなどのサービスを提供している.\hfill\cite{2015amazon}
    \end{block}
    \begin{minipage}{\textwidth}
        \centering
        \begin{minipage}[t]{.32\textwidth}
            \centering
            \includegraphics[keepaspectratio,width=\textwidth]{Arch_Amazon-EC2_64@5x.png}\\
            {\tiny Amazon Elastic Compute Cloud (Amazon EC2)\cite{aws}}
        \end{minipage}
        \begin{minipage}[t]{.32\textwidth}
            \centering
            \includegraphics[keepaspectratio,width=\textwidth]{Arch_Amazon-Simple-Storage-Service_64@5x.png}
            {\tiny Amazon Simple Storage Service (Amazon S3)\cite{aws}}
        \end{minipage}
        \begin{minipage}[t]{.32\textwidth}
            \centering
            \includegraphics[keepaspectratio,width=\textwidth]{Arch_Amazon-RDS_64@5x.png}\\
            {\tiny Amazon Relational Database Service (Amazon RDS)\cite{aws}}
        \end{minipage}
    \end{minipage}
\end{frame}
\subsection{オンプレミスとAWS}
\begin{frame}[t]{\ftitle}
    AWSはオンプレミスに比べてリソースを効率よく運用できる.\hfill\cite{2015amazon}
    \begin{figure}
        \centering
        \begin{minipage}[b]{.48\textwidth}
            \centering
            \begin{tikzpicture}
                \coordinate (O) at (0,0);
                \coordinate (A) at (1,2);
                \coordinate (B) at (2,1.5);
                \coordinate (C) at (3,3);
                \coordinate (D) at (4,2);
                \draw[red,thick, name path=curve]plot[smooth, tension=0.7]coordinates {(O)(A)(B)(C)(D)};
                \draw[name path=line,thick,dashed](0,2.5)--(4,2.5);
                \draw[thick,dashed](0,2.5)--(4,2.5);
                \draw[thick,-latex](0,0)--(4,0);
                \draw[thick,-latex](0,0)--(0,4.5);
                \node[below] at ($(0,0)!0.5!(4,0)$){\scriptsize 時間};
                \node[left] at ($(0,0)!0.5!(0,4.5)$){\scriptsize\rotatebox{90}{リソース量}};
                \node[below,text=red]at(B){\scriptsize 実際のリソース};
                \node[above]at(C){\scriptsize 機会損失};
                \node[below right]at(0,2.5){\scriptsize 無駄になる};
                \begin{scope}[on background layer]
                    \clip plot[smooth, tension=0.7] coordinates {(O) (A) (B) (C) (D)} |- (0,4) -- cycle;
                    \fill[yellow, name intersections={of=curve and line}] (intersection-1) rectangle (0,0);
                    \fill[yellow, name intersections={of=curve and line}] (intersection-2) rectangle (4,0);
                \end{scope}
            \end{tikzpicture}\\
            \vspace{1em}
            オンプレミス
        \end{minipage}
        \begin{minipage}[b]{.48\textwidth}
            \centering
            \begin{tikzpicture}
                \coordinate (O) at (0,0);
                \coordinate (A) at (1,2);
                \coordinate (B) at (2,1.5);
                \coordinate (C) at (3,3);
                \coordinate (D) at (4,2);
                \newcommand{\avv}{0.2cm}
                \draw[red,thick]plot[smooth, tension=0.7]coordinates {(O)(A)(B)(C)(D)};
                \draw[thick,dashed]plot[smooth,tension=0.7] coordinates {($(O)+(0,\avv)$)($(A)+(0,\avv)$)($(B)+(0,\avv)$)($(C)+(0,\avv)$)($(D)+(0,\avv)$)};
                \draw[thick,-latex](0,0)--(4,0);
                \draw[thick,-latex](0,0)--(0,4.5);
                \node[below] at ($(0,0)!0.5!(4,0)$){\scriptsize 時間};
                \node[left] at ($(0,0)!0.5!(0,4.5)$){\scriptsize\rotatebox{90}{リソース量}};
                \node[above]at($(C)+(0,\avv)$){\scriptsize 用意したリソース};
                \node[below,text=red]at(B){\scriptsize 必要なリソース};
            \end{tikzpicture}\\
            \vspace{1em}
            AWSの利用
        \end{minipage}
    \end{figure}
\end{frame}
\subsection{レンタルサーバとAWS}
\begin{frame}[t]{\ftitle}
    \begin{block}{レンタルサーバ}
        ``1台のサーバを複数のユーザが共用で利用する形態''\hfill\cite{2015amazon}
    \end{block}
    \begin{table}
        \centering
        レンタルサーバの利用形態\\
        \vspace{1em}
        \fontsize{5pt}{10pt}\selectfont
        \renewcommand{\arraystretch}{1.3}
        \newcommand{\hobo}{ほぼ}
        \begin{tabularx}{\textwidth}{cRRR}
                             & \multicolumn{1}{c}{共用サーバ} & \multicolumn{1}{c}{専用サーバ} & \multicolumn{1}{c}{仮装専用サーバ} \\
            \hline
            {\tiny 利用形態}     & {1台の物理サーバを分割して利用}         & {1台の物理サーバを占有}             & {1台の物理サーバ上に立てた仮想サーバを利用}     \\
            \hline
            {\tiny コスト}      & 安い                        & 高い                        & 中間                          \\
            \hline
            {\tiny カスタマイズ}   & \hobo 不可能                 & 可能                        & 可能                          \\
            \hline
            {\tiny セキュリティ}   & { コントロール不可}               & { 高くコントロール可能}             & { 高くコントロール可能}               \\
            \hline
            {\tiny 他のユーザの影響} & 強く受ける                     & 受けない                      & 多少受ける                       \\
            \hline
        \end{tabularx}
    \end{table}
    \hfill 引用\cite{2015amazon}
\end{frame}
\begin{frame}[t]{\ftitle}
    \begin{block}{AWS\ EC2}
        EC2は1台の物理マシン上に,複数の仮想コンピュータを立ち上げて使い,管理者権限(\texttt{root})を持ったアカウントを使用し,その仮想コンピュータ内のすべてを管理する.\hfill\cite{2015amazon}
    \end{block}
    「EC2」とレンタルサーバの「仮想専用サーバ」は似たサービス.しかし\dots
    \begin{exampleblock}{EC2の特徴}
        \begin{itemize}
            \item CPUやメモリを容易に変更できる.\hfill\tikz[remember picture]{\node(A){};}
            \item ディスクを動的に追加\footnote{インスタンスの再起動なしに変更すること.}できる.
                  \item[]\hfill 等々\dots
        \end{itemize}
    \end{exampleblock}
    \begin{tikzpicture}[remember picture,overlay]
        \node[left=-.2cm,rounded corners,fill=yellow,align=center,thick,draw]at($(A)+(0,.2cm)$){\scriptsize これまでの\\\scriptsize レンタルサーバにはない!};
    \end{tikzpicture}
\end{frame}
\subsection{AWS導入例}
\begin{frame}[t]{\ftitle}
    \begin{itemize}
        \item Cloud LaTeX(アカリク)
              \begin{itemize}
                  \item コンパイルなど(EC2)
                  \item プロジェクト,テンプレート管理(S3,MySQL)
              \end{itemize}
        \item 株式会社任天堂
              \begin{itemize}
                  \item 『マリオカート ツアー』のDBに,「Amazon Aurora」(RDBMS) を採用.
              \end{itemize}
        \item 株式会社東京海上日動火災保険
        \item JCB
        \item 盛岡市
        \item 浜松市
        \item SUBARU
    \end{itemize}
    \hfill\cite{導入事例}
\end{frame}