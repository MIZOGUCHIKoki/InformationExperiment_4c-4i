\section{AWS}
\tocc
\subsection{代表的なIaaS}
\begin{frame}[t]{\ftitle}
    \begin{itemize}
        \setlength{\itemsep}{1em}
        \item \underline{AWS}
        \item Microsoft Azure
        \item Google Cloud Pratform
        \item さくらのクラウド(さくらインターネット)
    \end{itemize}
\end{frame}
\subsection{AWS}
\begin{frame}[t]{\ftitle}
    \begin{block}{AWS}
        Amazon Web Service の略称.Amazonが提供するクラウドサービスで,ネットワークを経由して仮想コンピュータやストレージなどのサービスを提供している.\hfill\cite{2015amazon}
    \end{block}
    \begin{minipage}{\textwidth}
        \centering
        \begin{minipage}[t]{.32\textwidth}
            \centering
            \includegraphics[keepaspectratio,width=\textwidth]{Arch_Amazon-EC2_64@5x.png}\\
            {\tiny Amazon Elastic Compute Cloud (Amazon EC2)\cite{aws}}
        \end{minipage}
        \begin{minipage}[t]{.32\textwidth}
            \centering
            \includegraphics[keepaspectratio,width=\textwidth]{Arch_Amazon-Simple-Storage-Service_64@5x.png}
            {\tiny Amazon Simple Storage Service (Amazon S3)\cite{aws}}
        \end{minipage}
        \begin{minipage}[t]{.32\textwidth}
            \centering
            \includegraphics[keepaspectratio,width=\textwidth]{Arch_Amazon-RDS_64@5x.png}\\
            {\tiny Amazon Relational Database Service (Amazon RDS)\cite{aws}}
        \end{minipage}
    \end{minipage}
\end{frame}
\subsection{オンプレミスとAWS}
\begin{frame}[t]{\ftitle}

\end{frame}
\subsection{レンタルサーバとAWS}
\begin{frame}[t]{\ftitle}

\end{frame}
\subsection{AWS導入例}
\begin{frame}[t]{\ftitle}
    \begin{itemize}
        \item Cloud LaTeX(アカリク)
              \begin{itemize}
                  \item コンパイルなど(EC2)
                  \item プロジェクト,テンプレート管理(S3)
              \end{itemize}
        \item 任天堂(株)
              \begin{itemize}
                  \item 『マリオカート ツアー』のDBに,「Amazon Aurora」(RDMS) を採用.
              \end{itemize}
        \item 東京海上日動火災保険(株)
        \item JCB
        \item 盛岡市
        \item 浜松市
        \item SUBARU
    \end{itemize}
\end{frame}