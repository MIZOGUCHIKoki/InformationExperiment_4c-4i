\documentclass[aspectratio=43]{beamer}
\usepackage{luatexja}
\usepackage{hyperref,listings,tabularx,tikz,pifont}
\usetikzlibrary{arrows.meta,positioning,calc,fit,shapes.geometric}
\tikzset{str/.style={rounded corners, minimum height=1cm,align=center}}
\usepackage[symbol]{footmisc}
\newcommand{\cmark}{{\Large\bfseries\color{green!50!black}\ding{51}}}
\newcommand{\xmark}{{\Large\bfseries\color{red!50!black}\ding{55}}}
\hypersetup{
    colorlinks=true,
    linkcolor=black,
    urlcolor=blue
}
\usetheme{Montpellier}
\usecolortheme[RGB={160,82,45}]{structure}
\setbeamertemplate{navigation symbols}{}
\renewcommand{\familydefault}{\rmdefault}
\renewcommand{\figurename}{図\thefigure}
\renewcommand{\tablename}{表\thetable}
\renewcommand{\lstlistingname}{src.}
\renewcommand{\thefootnote}{\fnsymbol{footnote}}
\renewcommand{\thempfootnote}{\fnsymbol{mpfootnote}}
\bibliographystyle{junsrt}
\setbeamertemplate{bibliography item}[text]
\setbeamertemplate{section in toc}{
    {\color{structure}\inserttocsectionnumber.}\ \inserttocsection\vspace{.3em}
}

\setbeamertemplate{subsection in toc}{
    \hspace{1.2em}{\color{structure}\rule[.3ex]{3pt}{3pt}}\ \inserttocsubsection\par\vspace{.3em}
}
\setbeamertemplate{footline}{
    \leavevmode
    \hbox{
        \begin{beamercolorbox}[wd=.95\paperwidth,ht=1ex,right,dp=1.5ex,rightskip=0em]{bg=gray}
            \insertframenumber{} / \inserttotalframenumber
        \end{beamercolorbox}
    }
}
\setbeamercolor{block title}{use=structure,fg=white,bg=structure.fg!75!black}
\setbeamercolor{block title example}{parent=example text, fg=white,bg=example text.fg!75!black}
\setbeamercolor{block title alerted}{parent=alerted text, fg=white,bg=alerted text.fg!75!black}
\setbeamercolor{block body}{parent=normal text,use=block title,bg=block title.bg!10}
\setbeamercolor{block body alerted}{parent=normal text,use=block title alerted,bg=block title alerted.bg!10}
\setbeamercolor{block body example}{parent=normal text,use=block title example,bg=block title example.bg!10}
\setbeamertemplate{enumerate item}[default]
\setbeamertemplate{itemize item}[triange]
\newcolumntype{C}{>{\centering\arraybackslash}X}
\newcolumntype{R}{>{\raggedright\arraybackslash}X}
\newcolumntype{L}{>{\raggedleft\arraybackslash}X}
\lstdefinelanguage{docker-compose-2}{
    keywords={version, volumes, services},
    morekeywords={FROM,RUN,CMD,EXPOSE,ENV},
    keywords=[2]{image, environment, ports, container_name, ports, links, build},
    sensitive=false,
    comment=[l]{\#},
    morestring=[b]',
    morestring=[b]"
}
\lstset{
    %枠外に行った時の自動改行
    breaklines = true,
    %自動改行後のインデント量(デフォルトでは20[pt])
    breakindent = 10pt,
    %標準の書体
    basicstyle = \ttfamily\small,
    %コメントの書体
    commentstyle = {\scriptsize\ttfamily \color[cmyk]{1,0.4,1,0}},
    %関数名等の色の設定
    classoffset = 0,
    %キーワード(int, ifなど)の書体
    keywordstyle = {\bfseries\ttfamily \color[cmyk]{0,1,0,0}},
    %表示する文字の書体
    stringstyle = {\ttfamily \color[rgb]{0,0,1}},
    %枠 tは上に線を記載, Tは上に二重線を記載
    %他オプション:leftline,topline,bottomline,lines,single,shadowbox
    frame = single,
    %frameまでの間隔(行番号とプログラムの間)
    framesep = 5pt,
    %行番号の位置
    numbers = left,
    %行番号の間隔
    stepnumber = 1,
    %行番号の書体
    numberstyle = \scriptsize,
    %タブの大きさ
    tabsize = 4,
    %キャプションの場所(tbならば上下両方に記載)
    captionpos = t,
    % 左余白
    xleftmargin=1em,
}
\makeatletter
\@addtoreset{footnote}{page}
\makeatother
\newcommand{\ftitle}{\thesection.\thesubsection\ \subsecname}
\newcommand{\fftitle}{\thesection.\ \secname}
\newcommand{\stocsec}{{1-3}}
\newcommand{\ftocsec}{{4-6}}
\newcommand{\tocc}{
    \begin{frame}[t]{\thesection.\ \secname}
        \begin{columns}[t]
            \begin{column}{.48\textwidth}
                \tableofcontents[sections=\stocsec,currentsection,sectionstyle=show/shaded,subsectionstyle=show/show/shaded]
            \end{column}
            \begin{column}{.48\textwidth}
                \tableofcontents[sections=\ftocsec,currentsection,sectionstyle=show/shaded,subsectionstyle=show/show/shaded]
            \end{column}
        \end{columns}
    \end{frame}
}
\title{仮想化技術とクラウドコンピューティング}
\subtitle{AWS第1回}
\author{吉川 恭平\and 塩澤 康志\and 福島 康記\and 溝口 洸熙}
\institute{情報学群実験第4 Group C5}
\date{July 24th, 2023}
\begin{document}
\begin{frame}
    \maketitle
\end{frame}
\begin{frame}[t]{お品書き}
    \begin{columns}[t]
        \begin{column}{.48\textwidth}
            \tableofcontents[sections=\stocsec]
        \end{column}
        \begin{column}{.48\textwidth}
            \tableofcontents[sections=\ftocsec]
        \end{column}
    \end{columns}
\end{frame}
\section{仮想化技術と仮想マシン}
\tocc
\subsection{仮想化と仮想マシン}
\newcommand{\ie}{\tikz{\draw[arrows={-Latex[line width=1pt,fill=white,length=10pt]}](0,0)--(1,0)}\ }
\begin{frame}[t]{\ftitle}
    \newcommand{\server}{サーバー}
    \begin{block}{仮想化}
        ``コンピューターの物理的資源を論理的に分割して,それぞれ独立並列した状態で利用できるようにすること.
        1台の\server で,複数の基本ソフトを独立並列に動作させる\server 仮想化など.''\hfill\cite{スーパー大辞林}
    \end{block}
    \ie 1台の計算機で複数のOSやアプリケーションなどを並列に動作させること.
    \vspace{.5em}
    \begin{block}{Virtual Machine(仮想機械,仮想マシン)}
        ``あるコンピューターシステムの動作を,別システムで再現するソフトウエア.また,そのような動作環境.あるOSの動作を別のOS上で再現する場合など.バーチャルマシン.VM.''\hfill\cite{スーパー大辞林}
    \end{block}
    \ie 仮想化するためのソフトウェアや動作環境.
\end{frame}
\begin{frame}{\ftitle}
    \begin{figure}
        \centering
        \begin{minipage}[b]{.48\textwidth}
            \centering
            \begin{tikzpicture}
                \node[str,fill=gray!60,text width=.42\textwidth](hw1){物理サーバ};
                \node[str,fill=gray!40,text width=.42\textwidth,above=.1cm of hw1](os1){OS};
                \node[str,fill=gray!20,text width=.42\textwidth,above=.1cm of os1](sw1){\scriptsize アプリケーション};
                \node[str,fill=gray!60,text width=.42\textwidth,right=.1cm of hw1](hw2){物理サーバ};
                \node[str,fill=gray!40,text width=.42\textwidth,above=.1cm of hw2](os2){OS};
                \node[str,fill=gray!20,text width=.42\textwidth,above=.1cm of os2](sw2){\scriptsize アプリケーション};
                \node[inner sep=.05cm,draw,rounded corners,fit={(hw1)(os1)(sw1)}](warp1){};
                \node[inner sep=.05cm,draw,rounded corners,fit={(hw2)(os2)(sw2)}](warp2){};
            \end{tikzpicture}
            仮想化前
        \end{minipage}
        \begin{minipage}[b]{.48\textwidth}
            \centering
            \begin{tikzpicture}
                \node[str,fill=gray!60,text width=.85\textwidth](hw1){ハードウェア};
                \node[str,fill=gray!50,text width=.85\textwidth,above=.1cm of hw1](vmm){仮想化ソフトウェア};
                \node[str,fill=gray!40,text width=.38\textwidth,above=.1cm of vmm.north east,anchor=south east](os1){OS};
                \node[str,fill=gray!40,text width=.38\textwidth,above=.1cm of vmm.north west,anchor=south west](os2){OS};
                \node[str,fill=gray!20,text width=.38\textwidth,above=.1cm of os1](sw1){\tiny アプリケーション};
                \node[str,fill=gray!20,text width=.38\textwidth,above=.1cm of os2](sw2){\tiny アプリケーション};
                \node[inner sep=.05cm,draw,rounded corners,fit={(hw1)(vmm)},draw=white](warp1){};
            \end{tikzpicture}
            仮想化後
        \end{minipage}
    \end{figure}
\end{frame}
\subsection{仮想化の歴史と展望}
\begin{frame}{\ftitle}
    \begin{figure}
        \centering
        \begin{tikzpicture}
            \tikzset{txt/.style={rounded corners,draw,text width=.9\textwidth,align=center}};
            \node[txt](v1){2000年代の仮想化導入目的は\ \textbf{\color{red}コスト削減}};
            \newcommand{\ga}{が}
            \node[txt,below=1cm of v1](v2){仮想化の利用\ga 一巡する};
            \node[txt,below=1cm of v2](v3){本来持つ利点\ga 認識される};
            \node[txt,below=1cm of v3](v4){仮想化の\textbf{\color{red}俊敏性}・\textbf{\color{red}迅速性}・\textbf{\color{red}柔軟性}をビジネスに活かす};
            \foreach \u \v in {v1/v2,v2/v3,v3/v4}{
                    \draw[line width=.2cm,-latex](\u)--(\v);
                }
        \end{tikzpicture}
    \end{figure}
\end{frame}
\begin{frame}[t]{\ftitle}
    \begin{figure}
        \centering
        \begin{tikzpicture}
            \tikzset{rc/.style={text width=.3\textwidth,minimum height=2cm,rounded corners,align=center}};
            \tikzset{p/.style={align=center,rounded corners,fill=blue!60,minimum height=2cm,anchor=west,text=white,text width=.5cm,}};
            \tikzset{sfit/.style={draw=blue!60,inner sep=-.1mm,rounded corners}};
            \node[rc](p1){コスト削減};
            \node[p,left=0cm of p1.west](p){目\\的};
            \node[fit={(p)(p1)},sfit](pp1){};
            \node[above=.1cm of pp1](1t){\LARGE 一巡};
            \node[rc,below=.5cm of p1](a1){\small オンプレミスで構築された物理サーバを仮想サーバに集約};
            \node[p,left=0cm of a1](a){\rotatebox{90}{Action}};
            \node[fit={(a)(a1)},sfit](aa1){};
            \node[fit={(1t)(aa1)(pp1)},draw,rounded corners](warp1){};

            \node[rc,right=2.5cm of p1](p2){仮想化の俊敏性,迅速性,柔軟性をビジネスに活かす};
            \node[p,left=0cm of p2.west](p){目\\的};
            \node[fit={(p)(p2)},sfit](pp2){};
            \node[above=.1cm of pp2](2t){\LARGE 二巡};
            \node[rc,below=.5cm of p2,align=left](a2){\scriptsize ・システム開発の短縮化\\\scriptsize ・ITを駆使したビジネスのスピードアップ};
            \node[p,left=0cm of a2](a){\rotatebox{90}{Action}};
            \node[fit={(a)(a2)},sfit](aa2){};
            \node[fit={(2t)(aa2)(pp2)},draw,rounded corners](warp2){};
            \draw[-latex,line width=.2cm](warp1.east)--(warp2.west);
        \end{tikzpicture}
    \end{figure}
\end{frame}
\section{仮想化の方式}
\tocc
\subsection{ホスト型}
\begin{frame}[t]{\ftitle}
    ハードウェアの中のOS上に,土台となる仮想ソフトウェアをインストールし,仮想化ソフトウェアで仮想マシンを稼働させる.
    \begin{figure}[b]
        \centering
        \begin{tikzpicture}
            \node[str,fill=gray!60,text width=.9\textwidth](hw){ハードウェア};
            \node[str,fill=gray!50,text width=.9\textwidth,above=.1cm of hw](hos){ホストOS};
            \node[str,fill=gray!40,text width=.9\textwidth,above=.1cm of hos](hsw){ホスト型仮想化ソフトウェア};
            \node[str,fill=gray!30,text width=.43\textwidth,above=.1cm of hsw.north west,anchor=south west](gos1){ゲストOS};
            \node[str,fill=gray!30,text width=.43\textwidth,above=.1cm of hsw.north east,anchor=south east](gos2){ゲストOS};
            \node[str,fill=gray!20,text width=.2\textwidth,above=.1cm of gos1.north west,anchor=south west](sw1){\tiny アプリケーション};
            \node[str,fill=gray!20,text width=.2\textwidth,above=.1cm of gos1.north east,anchor=south east](sw2){\tiny アプリケーション};
            \node[str,fill=gray!20,text width=.2\textwidth,above=.1cm of gos2.north west,anchor=south west](sw3){\tiny アプリケーション};
            \node[str,fill=gray!20,text width=.2\textwidth,above=.1cm of gos2.north east,anchor=south east](sw4){\tiny アプリケーション};
        \end{tikzpicture}
    \end{figure}
    % \hyperlink{ハイパーバイザ型}{\beamergotobutton{ハイパーバイザ型}}\hypertarget{ホスト型}{}
\end{frame}
\begin{frame}[t]{\ftitle}
    \begin{exampleblock}{ホスト型仮想化ソフトウェア 例}
        \begin{minipage}[b]{.8\textwidth}
            \begin{itemize}
                \setlength{\itemsep}{1em}
                \item VMware Workstation Player
                \item VMware Fusion
                \item Oracle VM Virtualbox
            \end{itemize}
        \end{minipage}
        \begin{minipage}[b]{.15\textwidth}
            \centering
            \includegraphics[keepaspectratio,width=.8\textwidth]{virtualbox_logo.png}\\
            {\tiny Virtualbox\cite{VMBox}}
        \end{minipage}
    \end{exampleblock}
    \begin{minipage}[t]{.48\textwidth}
        \textbf{\cmark メリット}
        \begin{itemize}
            \item 既存マシンが利用できる点.
            \item 仮想化に必要なソフトウェアが扱いやすい.
        \end{itemize}
    \end{minipage}
    \begin{minipage}[t]{.48\textwidth}
        \textbf{\xmark デメリット}
        \begin{itemize}
            \item ホストOSを動作させるための物理リソースが必要.
        \end{itemize}
    \end{minipage}\\
    \hfill\cite{itmanage}
\end{frame}
\subsection{ハイパーバイザ型}
\begin{frame}[t]{\ftitle}
    ハイパーバイザとは「仮想化のためのOS」のようなもの.ホストOSを必要としない,仮想化ソフトウェア.
    \begin{figure}[b]
        \centering
        \begin{tikzpicture}
            \node[str,fill=gray!60,text width=.9\textwidth](hw){ハードウェア};
            \node[str,fill=gray!45,text width=.9\textwidth,above=.1cm of hw,minimum height=2.1cm](hpv){ハイパーバイザ};
            \node[str,fill=gray!30,text width=.43\textwidth,above=.1cm of hpv.north west,anchor=south west](gos1){ゲストOS};
            \node[str,fill=gray!30,text width=.43\textwidth,above=.1cm of hpv.north east,anchor=south east](gos2){Linux};
            \node[str,fill=gray!20,text width=.2\textwidth,above=.1cm of gos1.north west,anchor=south west](sw1){\tiny アプリケーション};
            \node[str,fill=gray!20,text width=.2\textwidth,above=.1cm of gos1.north east,anchor=south east](sw2){\tiny アプリケーション};
            \node[str,fill=gray!20,text width=.2\textwidth,above=.1cm of gos2.north west,anchor=south west](sw3){Apache};
            \node[str,fill=gray!20,text width=.2\textwidth,above=.1cm of gos2.north east,anchor=south east](sw4){Python};
        \end{tikzpicture}
    \end{figure}
    % \hyperlink{ホスト型}{\beamergotobutton{ホスト型}}\hypertarget{ハイパーバイザ型}{}
\end{frame}
\begin{frame}[t]{\ftitle}
    \begin{exampleblock}{ハイパーバイザ 例}
        \begin{itemize}
            \item VMware ESXi
            \item Linux KVM
            \item Microsoft Hyper-V
        \end{itemize}
    \end{exampleblock}
    \begin{minipage}[t]{.48\textwidth}
        \textbf{\cmark メリット}
        \begin{itemize}
            \item ホスト型に比べて,システム全体の観点から見てリソースの使用効率がよい.
            \item 物理サーバに比べて,運用にかかるコストを削減できる.
        \end{itemize}
    \end{minipage}
    \begin{minipage}[t]{.48\textwidth}
        \textbf{\xmark デメリット}
        \begin{itemize}
            \item 物理サーバに比べて,性能が劣る.
            \item 物理サーバに比べて,障害の範囲は大きくなる(ことがある.)
        \end{itemize}
    \end{minipage}\\
    \hfill\cite{itmanage}
\end{frame}
\subsection{コンテナ型}
\begin{frame}[t]{\ftitle}
    ``アプリケーションを実行するための領域(ユーザ空間)を複数に分割して利用するもの''\cite{itmanage}.
    \begin{figure}[b]
        \centering
        \begin{tikzpicture}
            \node[str,fill=gray!60,text width=.9\textwidth](hw){ハードウェア};
            \node[str,fill=gray!50,text width=.9\textwidth,above=.1cm of hw](hos){ホストOS};
            \node[str,fill=gray!40,text width=.9\textwidth,above=.1cm of hos](hsw){Docker または Kubernetes};
            \node[str,fill=gray!30,text width=.2\textwidth,above=.1cm of hsw.north west,anchor=south west](gos1){ミドルウェア};
            \node[str,fill=gray!30,text width=.2\textwidth,above=.1cm of hsw.north east,anchor=south east](gos2){DBMS};
            \node[str,fill=gray!20,text width=.2\textwidth,above=.1cm of gos1.north west,anchor=south west](sw1){\tiny アプリケーション};
            \node[str,fill=gray!20,text width=.2\textwidth,above=.1cm of gos2.north east,anchor=south east](sw4){MySQL};
            \node[inner sep=.5mm,fit={(gos1)(sw1)},draw,thick,dotted,rounded corners](wrap1){};
            \node[inner sep=.5mm,fit={(gos2)(sw4)},draw,thick,dotted,rounded corners](wrap2){};
            \node at ($(wrap1)!0.5!(wrap2)+(0,.5cm)$)(cont){コンテナ};
            \draw[-latex](cont.west)--(wrap1.east);
            \draw[-latex](cont.east)--(wrap2.west);
        \end{tikzpicture}
    \end{figure}
\end{frame}
\begin{frame}[t]{\ftitle}
    \begin{block}{ミドルウェア}
        システムソフトウェアは,基本ソフトウェア(OS)とミドルウェアに分類される.
        ミドルウェアはOSとアプリケーションソフトウェアの中立ちをする.\hfill\cite{ITの基礎}
    \end{block}
    \begin{exampleblock}{ミドルウェアの例}
        データベースを管理する,データベースを操作する基本ソフトウェア(DBMS\footnote{Data Base Management System})がある.
    \end{exampleblock}
\end{frame}
\begin{frame}[t]{\ftitle}
    \begin{minipage}[t]{.48\textwidth}
        \textbf{\cmark メリット}
        \begin{itemize}
            \item バージョン依存が激しいもの(Python,JavaScript)などでも指定した環境を再現しやすい.(\texttt{Dockerfile})
            \item 環境構築が簡単.
            \item ゲストOSが不要なので,アプリケーションの起動や処理が高速.
        \end{itemize}
    \end{minipage}
    \begin{minipage}[t]{.48\textwidth}
        \textbf{\xmark デメリット}
        \begin{itemize}
            \item 同一基盤上で異なるOSを動かせない.
            \item ホストOSで障害が生じると,すべてのコンテナに影響が出る.
        \end{itemize}
    \end{minipage}
\end{frame}
\section{Docker(Option)}\label{chap:docker}
\tocc
\subsection{Docker 利用手順}
\begin{frame}[t]{\ftitle}
    \begin{itemize}
        \setlength{\itemsep}{1em}
        \item コンテナを利用する.
              \begin{enumerate}
                  \setlength{\itemsep}{.5em}
                  \item イメージを作成(\texttt{build})または,取得(\texttt{docker pull})する.\\
                        イメージの作成には\texttt{Dockerfile} を書く.
                  \item コンテナを作成して起動(\texttt{run})する.
                  \item コンテナ内でコマンドを実行(\texttt{exec})する.
              \end{enumerate}
        \item コンテナ利用後.
              \begin{enumerate}
                  \setlength{\itemsep}{.5em}
                  \item コンテナを停止(\texttt{stop})する.
                  \item コンテナを削除(\texttt{rm})する.
                  \item イメージを削除(\texttt{rmi})する.
              \end{enumerate}
    \end{itemize}
\end{frame}
\subsection{\texttt{Dockerfile}}
\begin{frame}[containsverbatim,t]{\ftitle}
    \begin{lstlisting}[language=docker-compose-2]
FROM: ubuntu:latest
RUN apt-get update # イメージ作成時
CMD ["ping","-c","3","1.1.1.1"] # コンテナ作成時
EXPOSE 3306 # ポートの解放
\end{lstlisting}
\end{frame}
\subsection{実演}
\begin{frame}{\ftitle}
    \includegraphics[keepaspectratio,width=\textwidth]{docker_logo.png}\\\vspace{1em}
    \hfill\cite{docker}
\end{frame}
\section{準仮想化と完全仮想化}\hypertarget{sec:準仮想化と完全仮想化}{}
\tocc
\begin{frame}[t]{\fftitle}
    ホストOSとハイパーバイザ型を比較すると,性能的にはハイパーバイザ型が有利.
    ハイパーバイザの中にも\textbf{完全仮想化}と\textbf{準仮想化}がある.
    \begin{block}{完全仮想化}
        ハードウェアも含めて,すべてを仮想化する方式.ハードウェアも論理ハードウェア.
    \end{block}
    \begin{block}{準仮想化}
        ゲストOSを少し改造して,ゲストOSから直接ハードウェアを操作できる方式.
        処理速度が大きくなる.(理由は次頁.)
    \end{block}
    \hfill\hyperlink{sec:クラウドコンピューティング}{\beamerreturnbutton{詳しい説明を飛ばす}}
\end{frame}
\begin{frame}[t]{\fftitle (詳しく)}
    \begin{block}{完全仮想化}
        ハイパーバイザ上で動作するゲストOSの中身に何も変更を加えないで動作させる方式.
        ゲストOSからハードウェアを操作する(特権命令の実行)は許されないので,ハイパーバイザで処理を変換する.
    \end{block}
    \begin{block}{準仮想化}
        ゲストOSに変更を加えて性能向上を図った方式.
        ゲストOSからハードウェアを操作する部分(特権命令)を,ハイパーバイザに直接処理が渡るように変更を加える.
        ハイパーバイザでの命令変換時にかかるオーバーヘッドを削減する
    \end{block}
    \hfill\cite{仮想化技術}
\end{frame}
\begin{frame}[t]{\fftitle (詳しく)}
    \newcommand{\karano}{からの}
    命令\texttt{A}:ゲストOSからの特権命令\footnote{CPUのステータスを変更したり,オペレーティングシステムを動作させるための命令.}.\\
    命令\texttt{X}:ハイパーバイザからの特権命令.
    \begin{figure}
        \centering
        \begin{tikzpicture}
            \node[draw,fill=gray!60,minimum height=1cm,align=center,text width=.4\textwidth](vm1){命令\texttt{A}};
            \node[draw,fill=gray!40,minimum height=1cm,align=center,text width=.4\textwidth,below=.5cm of vm1](vmm1){命令\texttt{A}\ \raisebox{.25\baselineskip}{\tikz{\draw[-latex](0,0)--(.5,0)}}\ \texttt{X}};
            \node[draw,fill=gray!20,minimum height=1cm,align=center,text width=.4\textwidth,below=.5cm of vmm1](hw1){命令\texttt{X}};
            \node[anchor=north west,draw,fill=white] at (vm1.north west){\tiny ゲストOS};
            \node[anchor=north west,draw,fill=white] at (vmm1.north west){\fontsize{4pt}{0pt}\selectfont ハイパーバイザ};
            \node[anchor=north west,draw,fill=white] at (hw1.north west){\tiny ハードウェア};
            \node[draw,fill=gray!60,minimum height=1cm,align=center,text width=.4\textwidth,right=.5cm of vm1](vm2){命令\texttt{X}};
            \node[draw,fill=gray!40,minimum height=1cm,align=center,text width=.4\textwidth,below=.5cm of vm2](vmm2){命令\texttt{X}};
            \node[draw,fill=gray!20,minimum height=1cm,align=center,text width=.4\textwidth,below=.5cm of vmm2](hw2){命令\texttt{X}};
            \node[anchor=north west,draw,fill=white] at (vm2.north west){\tiny ゲストOS};
            \node[anchor=north west,draw,fill=white] at (vmm2.north west){\fontsize{4pt}{0pt}\selectfont ハイパーバイザ};
            \node[anchor=north west,draw,fill=white] at (hw2.north west){\tiny ハードウェア};
            \node[below] at (hw1.south){完全仮想化};
            \node[below] at (hw2.south){準仮想化};
            \foreach \u \v in {vm1/vmm1,vmm1/hw1,vm2/vmm2,vmm2/hw2}{
                    \draw[-latex](\u)--(\v);
                }
            \node[align=center,draw,fill=gray!60,rounded corners] at($(vm2.west)+(.6cm,-.5cm)$) {\tiny あらかじめ特権命\\\tiny 令を変換しておく};
            \node[align=center,draw,fill=gray!40,rounded corners] at($(vmm1.east)+(-.8cm,-.5cm)$) {\tiny 特権命令を変換する};
        \end{tikzpicture}
    \end{figure}
\end{frame}
\section{クラウドコンピューティング}
\subsection{クラウドとは}
\begin{frame}[t]{\ftitle}
    \begin{block}{クラウドコンピューティング}
        クラウドコンピューティング(以下,クラウド)とは,コンピュータリソース尾の利用形態.
        コンピュータの計算リソースやストレージ領域,アプリケーションによる処理をネットワーク経由で提供する.\hfill\cite{2015amazon}
    \end{block}
    クラウドは以下の4つに分類されることが多い.
    \begin{itemize}
        \item IaaS(Insfrastructure as a Service)
        \item PaaS(Platform as a Service)
        \item SaaS(Software as a Service)
        \item DaaS(Desktop as a Service)% VDIとの違い
    \end{itemize}
\end{frame}
\section{AWS}
\tocc
\subsection{代表的なIaaS}
\begin{frame}[t]{\ftitle}
    \begin{itemize}
        \setlength{\itemsep}{1em}
        \item \underline{AWS}
        \item Microsoft Azure
        \item Google Cloud Pratform
        \item さくらのクラウド(さくらインターネット)
    \end{itemize}
\end{frame}
\subsection{AWS}
\begin{frame}[t]{\ftitle}
    \begin{block}{AWS}
        Amazon Web Service の略称.Amazonが提供するクラウドサービスで,ネットワークを経由して仮想コンピュータやストレージなどのサービスを提供している.\hfill\cite{2015amazon}
    \end{block}
    \begin{minipage}{\textwidth}
        \centering
        \begin{minipage}[t]{.32\textwidth}
            \centering
            \includegraphics[keepaspectratio,width=\textwidth]{Arch_Amazon-EC2_64@5x.png}\\
            {\tiny Amazon Elastic Compute Cloud (Amazon EC2)\cite{aws}}
        \end{minipage}
        \begin{minipage}[t]{.32\textwidth}
            \centering
            \includegraphics[keepaspectratio,width=\textwidth]{Arch_Amazon-Simple-Storage-Service_64@5x.png}
            {\tiny Amazon Simple Storage Service (Amazon S3)\cite{aws}}
        \end{minipage}
        \begin{minipage}[t]{.32\textwidth}
            \centering
            \includegraphics[keepaspectratio,width=\textwidth]{Arch_Amazon-RDS_64@5x.png}\\
            {\tiny Amazon Relational Database Service (Amazon RDS)\cite{aws}}
        \end{minipage}
    \end{minipage}
\end{frame}
\subsection{オンプレミスとAWS}
\begin{frame}[t]{\ftitle}

\end{frame}
\subsection{レンタルサーバとAWS}
\begin{frame}[t]{\ftitle}

\end{frame}
\subsection{AWS導入例}
\begin{frame}[t]{\ftitle}
    \begin{itemize}
        \item Cloud LaTeX(アカリク)
              \begin{itemize}
                  \item コンパイルなど(EC2)
                  \item プロジェクト,テンプレート管理(S3)
              \end{itemize}
        \item 任天堂(株)
              \begin{itemize}
                  \item 『マリオカート ツアー』のDBに,「Amazon Aurora」(RDMS) を採用.
              \end{itemize}
        \item 東京海上日動火災保険(株)
        \item JCB
        \item 盛岡市
        \item 浜松市
        \item SUBARU
    \end{itemize}
\end{frame}
\begin{frame}[allowframebreaks]{Reference}
    \bibliography{bib}
\end{frame}
\end{document}